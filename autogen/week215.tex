
% </A>
% </A>
% </A>
\week{May 9, 2005 }

This week I'd like to report on some cool things people have been
explaining to me.  The science fiction writer Greg Egan has been 
helping me understand Klein's quartic curve, and the mathematician
Darin Brown has been explaining the analogy between geodesics and 
prime numbers.  The two subjects even overlap slightly!

Last week I talked about Klein's quartic curve.  This led
Gerard Westendorp and Mike Stay to draw some pictures of it, 
and their ideas helped Greg Egan create this really nice picture:

1) Greg Egan, Klein's quartic curve,
<A HREF = "http://math.ucr.edu/home/baez/mathematical/KleinDual.png">
http://math.ucr.edu/home/baez/mathematical/KleinDual.png</A>

<DIV ALIGN = CENTER>
<IMG SRC = "http://math.ucr.edu/home/baez/mathematical/KleinDual.png">
</DIV>


It looks sort of tetrahedral at first glance, but if you look 
carefully you'll see that topologically speaking, it's a 3-holed
torus.  It's tiled by triangles, with 7 meeting at each vertex.  
So, it's the Klein quartic curve!

Perhaps I should explain.  Last week I talked about a tiling of the 
hyperbolic plane by regular heptagons with 3 heptagons meeting at 
each vertex.  Dual to this is a tiling of the hyperbolic plane by 
equilateral triangles with 7 triangles meeting at each vertex.  
We can take a quotient space of this by a certain symmetry group 
and get a 3-holed torus tiled by 56 triangles with 7 meeting at each
vertex.  This is what Egan drew!

With this picture you can almost \emph{see} the 168 symmetries of Klein's 
quartic curve.

First, you can take any vertex and twist it, causing the 7 triangles
that meet at this vertex to cycle around.  It's not obvious that this
is a symmetry of the whole tiled surface, but it is.  This gives a 
7-element symmetry group.  

Second, the whole thing looks like a tetrahedron, so it inherits the
rotational symmetries of a tetrahedron.  This gives a more obvious
12-element symmetry group.  

7 \times  12 = 84, so how do we get a total of 168 symmetries?

Well, there's also a 2-fold symmetry that corresponds to turning 
the tetrahedron inside out!  And Egan made a wonderful \emph{movie} of 
this.  If a picture is worth a thousand words, this is worth about 
a million:

2) Greg Egan, Turning Klein's quartic curve inside out,
<A HREF = "http://math.ucr.edu/home/baez/mathematical/KleinDualInsideOut.gif">
http://math.ucr.edu/home/baez/mathematical/KleinDualInsideOut.gif</A>


<DIV ALIGN = CENTER>
<IMG SRC = "http://math.ucr.edu/home/baez/mathematical/KleinDualInsideOut.gif">
</DIV>
So, we get a total of 7 \times  24 = 168 symmetries. 

Even better, if you watch carefully, you'll see that the tetrahedron 
in Egan's movie gets \emph{reflected} as it turns inside out.  More 
precisely, if you follow the four corners of the tetrahedron, 
you'll see that two come back to where they were, while the other 
two get switched.  So, this symmetry acts as a reflection, or odd 
permutation, of the 4 corners.  The rotations act as even permutations 
of the corners.

This means that the Klein quartic has 24 symmetries forming a group
isomorphic to the rotation/reflection symmetry group of a tetrahedron.
Algebraically speaking, this group is S_{4}: 
the permutations of 4 things.

This group is also the rotational symmetry group of a cube.  In fact, 
Egan was able to spot a hidden cube lurking in his picture!  Can you?

<DIV ALIGN = CENTER>
<IMG HEIGHT = 300 WIDTH = 300 
SRC = "http://math.ucr.edu/home/baez/mathematical/KleinDual.png">
</DIV>

If you look carefully, you'll see each corner of his tetrahedral
gadget is made of a little triangular prism with one triangle facing 
out and one facing in: for example, the pink triangle
staring you right in the face, or the light blue one on top.
Since 4 \times  2 = 8, there are 8 of these triangles.  
Abstractly, we can think of these as the 8 corners of a cube!  They 
aren't really, but we can pretend.  The way these 8 triangles come 
in pairs corresponds to how the vertices of a cube come in diagonally 
opposite pairs.  

Using this, you can see that the group S_{4} acts on these 8 triangles 
in precisely the same way it acts via rotations on the vertices of a 
cube.

In fact, you can even draw a \emph{picture} of a cube on the Klein quartic by 
drawing suitable curves that connect the centers of these 8 triangles!
It's horribly distorted, but topologically correct.  Part of the 
distortion is caused by embedding the Klein quartic in ordinary
3d Euclidean space.  If we gave the Klein quartic the metric it 
inherits from the hyperbolic plane, the edges of the cube would be 
geodesics.

This remark also helps us see something else.  The Klein quartic is
tiled by 56 triangles.  8 of them give the cube we've just been 
discussing.  In Egan's picture these triangles look special, since 
they lie at the corners of his tetrahedral gadget.  But this is
just an illusion caused by embedding the Klein quartic in 3d space.  
In reality, the Klein quartic is perfectly symmetrical: every 
triangle is just like every other.  So in fact there are lots of 
these cubes, and every triangle lies in some cube.

But this is where it gets really cool.  In fact, each triangle lies
in just \emph{one} cube.  So, there's precisely one way to take the 56 
triangles and divide them into 7 bunches of 8 so that each bunch forms 
a cube.   

So: the symmetry group of the Klein quartic acts on the set of cubes,
which has 7 elements.

But as I explained last week, this symmetry group also acts on the 
Fano plane, which has 7 points.

This suggests that cubes in the Klein quartic naturally correspond 
to points of the Fano plane.  And Egan showed this is true!

He showed this by showing more.  The Fano plane also has 7 lines.
What 7 things in the Klein quartic do these lines correspond to?

\emph{Anticubes!}

You see, the cubes in the Klein quartic have an inherent handedness
to them.  You can go between the 8 triangles of a given cube by 
following certain driving directions, but these driving directions 
involve some left and right turns.  If you follow the mirror-image
driving directions with "left" and "right" switched, 
you'll get an \emph{anticube}.
 
Apart from having the opposite handedness, anticubes are just like
cubes.  In particular, there's precisely one way to take the 56 
triangles and divide them into 7 bunches of 8 so that each bunch forms 
an anticube.

Here's a picture:

3) Greg Egan, Cubes and anticubes in the Klein quartic curve,
<A HREF = "http://math.ucr.edu/home/baez/KleinFigures.gif">
http://math.ucr.edu/home/baez/KleinFigures.gif</A>

<BR>
<DIV ALIGN = CENTER>
<IMG SRC ="http://math.ucr.edu/home/baez/KleinFigures.gif">
</DIV>
<BR>

Each triangle has a colored circle and a colored square on it.
There are 7 colors.  The colored circle says which of the 7 \emph{cubes}
the triangle belongs to.  The colored square says which of the 7
\emph{anticubes} it belongs to.  

If you stare at this picture for a few hours, you'll see that each 
cube is completely disjoint from precisely 3 anticubes.  Similarly, 
each anticube is completely disjoint from precisely 3 cubes.  

This is just like the Fano plane, where each point lies on 3 lines, 
and each line contains 3 points!

So, we get a vivid way of seeing how every figure in the Fano 
plane corresponds to some figure in the Klein quartic curve. 
This is why they have the same symmetry group.  

This is an excellent example of Klein's Erlangen program for reducing 
geometry to group theory, which I discussed in "<A HREF = "week213.html">week213</A>".  Here we are 
beginning to see how two superficially different geometries are secretly 
the same:

\begin{verbatim}
    FANO PLANE                      KLEIN'S QUARTIC CURVE
 
    7 points                        7 cubes
    7 lines                         7 anticubes
    incidence of points and lines   disjointness of cubes and anticubes
\end{verbatim}
    
However, we're only half done!  We've seen how to translate simple 
figures and indicence relations in the Fano plane to complicated ones
in Klein's quartic curve.  But, we haven't figured out translate back!
 
\begin{verbatim}
    KLEIN'S QUARTIC CURVE            FANO PLANE
       
    24 vertices                      ??? 
    84 edges                         ???
    56 triangular faces              ???
    incidence of vertices and edges  ???
    incidence of edges and faces     ???
\end{verbatim}
    
Here I'm talking about the tiling of Klein's quartic curve by 56
equilateral triangles.  We could equally well talk about its tiling
by 24 regular heptagons, which is the Poincare dual.  Either way, the
puzzle is to fill in the question marks.  I don't know the answer!

To conclude - at least for now - I want to give the driving directions 
that define a "cube" or an "anticube" in Klein's quartic curve.  Say 
you're on some triangle and you want to get to a nearby triangle that 
belongs to the same cube.  Here's what you do:

\begin{quote}
hop across any edge, <br>
turn right, <br>
hop across the edge in front of you, <br>
turn left, <br>
then hop across the edge in front of you.  

\end{quote}
    
Or, suppose you're on some triangle and you want to get to another
that's in the same anticube.  Here's what you do:

\begin{quote}
hop across any edge, <br>
turn left, <br>
hop across the edge in front of you, <br>
turn right, <br>
then hop across the edge in front of you.  
 
\end{quote}
    
(If you don't understand this stuff, look at the picture above and 
see how to get from any circle or square to any other circle or 
square of the same color.) 

You'll notice that these instructions are mirror-image versions of 
each other.  They're also both 1/4 of the "driving directions from 
hell" that I described last time.  In other words, if you go 
LRLRLRLR or RLRLRLRL, you wind up at the same triangle you started
from.  You'll have circled around one face of a cube or anticube!

In fact, your path will be a closed geodesic on the Klein quartic 
curve... like the long dashed line in Klein and Fricke's original 
picture:

4) Klein and Fricke, Klein's quartic curve with geodesic,
<A HREF = "http://math.ucr.edu/home/baez/Klein168.gif">http://math.ucr.edu/home/baez/Klein168.gif</A>

<DIV ALIGN = CENTER>
<IMG SRC = "http://math.ucr.edu/home/baez/Klein168.gif">
</DIV>
Next, a little about geodesics and prime numbers.   I've just been
talking a little about geodesics in the Klein quartic, which is the
quotient 

H/G

of the hyperbolic plane H by a certain group G which I explained  
last week.  This group, usually called \Gamma (7), is a nice example 
of a "Fuchsian group" - that is, a discrete subgroup of the isometries
of the hyperbolic plane.  

Darin Brown and his thesis advisor Jeff Stopple at U. C. Santa
Barbara have been thinking about geodesics in H/G for other Fuchsian
groups G, and their relation to number theory:

5) Jeff Stopple, A reciprocity law for prime geodesics, J. Number 
Theory 29 (1988), 224-230.

6) Darin Brown, Lifting properties of prime geodesics on hyperbolic
surfaces, Ph.D. thesis, U. C. Santa Barbara, 2004.

I'd really like to learn about this, because it connects all sorts
of stuff I dream of understanding someday, especially quantum chaos 
("<A HREF = "week190.html">week190</A>"), zeta functions in physics and number theory ("<A HREF = "week199.html">week199</A>"), 
and Galois theory as a theory of covering spaces ("<A HREF = "week205.html">week205</A>").  Also, 
it involves a big mysterious analogy, and I always like those!

I don't understand this stuff well enough to try a full-fledged
explanation yet, so I'll just give a vague sketch.  A "prime geodesic"
in a Riemannian manifold X is a closed geodesic 

f: S^{1} \to  X

that cycles around just once.  In other words, f should be one-to-one.

We say a closed geodesic is the "nth power" of a prime one if it's
just like the prime one but it cycles around n times.  Every closed
geodesic is the nth power of a prime one in a unique way.

If we have a Fuchsian group G, H/G is a surface with a Riemannian
metric.  It looks locally like the hyperbolic plane, so it's called
a "hyperbolic surface".  And, we can look at prime geodesics in it.  

If G' is a subgroup of G, we get a covering map

H/G' \to  H/G

so we can ask about lifting prime geodesics in H/G to closed geodesics
in H/G'.   There can be a bunch of ways to do this, so we say a
prime geodesic in H/G "splits" into powers of prime geodesics up in
H/G'.  

If you know any number theory - reading "<A HREF = "week205.html">week205</A>" should be enough -
this should remind you of how a prime ideal in some algebraic number 
field can "split" into prime ideals in an extension of this field, 
and/or "ramify" into powers of prime ideals.

And indeed, Darin Brown has found a big mysterious analogy that goes 
like this:



% parser failed at source line 397
