
% </A>
% </A>
% </A>
\week{July 5, 2000 }


I've been reading about the mathematical physicist William Rowan 
Hamilton lately, because I'm writing a review article about the 
octonions - that famous nonassociative 8-dimensional division 
algebra.  

You see, the day after Hamilton discovered the quaternions and carved 
the crucial formula

$$
                i^{2} = j^{2} = k^{2} = ijk = -1
$$
    
on the Brougham bridge, he mailed a letter explaining his discovery
to his friend John Graves.  And about two months later, Graves 
discovered the octonions!  In December 1843, he sent a letter about 
them to Hamilton.   

Graves called them "octaves" at first, but later introduced
the term "octonions".  He showed they were a normed division
algebra and used this to prove the 8 squares theorem, which says that
the product of two sums of 8 perfect squares is again a sum of 8 perfect
squares.  The complex numbers and quaternions allow one to prove similar
theorems for 2 and 4 squares.  In January 1844, Graves considered the
idea of a general theory of "2^{m}-ions".  He tried 
to construct
a 16-dimensional normed division algebra and use it to prove a 16
squares theorem, but he "met with an unexpected hitch" and came to
doubt that this was possible.  

(If you read "<A HREF = "week59.html">week59</A>" you'll see why.)

Hamilton was the one who noticed that the octonions were nonassociative - in 
fact, he invented the word "associative" right about this time.  He 
offered to write a paper publicizing Graves' work, and Graves accepted 
the offer, but Hamilton kept putting it off.  He was probably busy 
working on the quaternions!  

Meanwhile, Arthur Cayley had heard about the quaternions right when 
Hamilton announced his discovery, and he eventually discovered the 
octonions on his own.  He published a description of them in the March 
1845 issue of the Philosophical Magazine.  Graves was upset, so he added 
a postscript about the octonions to a paper of his that was due to 
appear in the following issue of the same journal, asserting that 
he'd known about them since Christmas 1843.  Also, Hamilton eventually 
got his act together and published a short note about Graves' discovery 
in the June 1847 issue of the Proceedings of the Royal Irish Academy.  
But by then it was too late - everyone was calling the octonions "Cayley 
numbers".

Of course it wasn't \emph{really} too late, since everybody who cares 
can now tell that Graves was the first to discover the octonions.  
And anyway, it doesn't really make a difference who discovered them 
first, except as a matter of historical interest.  

But just for the heck of it, I'm trying to find out everything I can 
about the early history of the octonions.  Hamilton is very famous, and 
much has been written about him, but Graves is mainly famous for being 
Hamilton's friend - so to learn stuff about Graves, I have to read books 
on Hamilton.   In the process, I've learned some interesting things that 
aren't really relevant to my review article.  And I want to tell you about 
some of them before I forget! 

Hamilton was a strangely dreamy sort of guy.  He spent most of his
life as the head of a small observatory near Dublin, but quickly lost
interest in actually staying up nights to make observations.  Instead,
he preferred writing poetry.  He was friends with Coleridge, who 
introduced him to the philosophy of Kant, which influenced him greatly.
He was also friends with Wordsworth - who told him to not to write poetry.  
He fell deeply in love with a woman named Catherine Disney, who was forced 
by her parents to marry a wealthy man 15 years older than her.  Hamilton
remained hopelessly in love with her the rest of his life, though he 
eventually married someone else.  He became an alcoholic, then foreswore 
drink, then relapsed.  Eventually, many years later, Catherine began a 
secret correspondence with him - she still loved him!  Her husband became 
suspicious, she attempted suicide by taking laudanum... and then, five 
years later, she became ill.   Hamilton visited her and gave her a copy of 
his "Lectures on Quaternions" - they kissed at long last - and 
then she died two weeks later.  He carried her picture with him ever 
afterwards and talked about her to anyone who would listen.   A very 
sad and very Victorian tale.  

He was a bit too far ahead of his time to have maximum impact during his 
own life.  The Hamiltonian approach to mechanics and the Hamilton-Jacobi 
equation relating waves and particles became really important only when 
quantum mechanics came along.  Luckily Klein liked this stuff, and told 
Schroedinger about it.  But it's a pity that Hamilton's unification of 
particle and wave mechanics came along right when the advocates of the
wave theory of light seemed to have definitively won the battle against 
the particle theory - the need for a compromise became clear only later.  

Quaternions, too, might have had more impact if they'd come along later,
when people were trying to understand spin-1/2 particles.  After all, 
the unit quaternions form the group SU(2), which is perfect for studying
spin-1/2 particles.   But the way things actually went, quaternions were 
not very popular by the time people dreamt of spin-1/2 particles - so 
Pauli just used 2 x 2 complex matrices to describe the generators of SU(2).   

I like what Hamilton wrote about quaternions, space, and time:



% parser failed at source line 144
