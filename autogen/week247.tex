
% </A>
% </A>
% </A>
\week{March 23, 2007 }

Symmetry has fascinated us throughout the ages.  Greek settlers 
in Sicily may have seen irregular 12-sided crystals of pyrite in Sicily
and dreamt up the regular dodecahedron simply because it was more 
beautiful, more symmetrical. 

<div align = "center">
<a href = "http://www.minerals.net/mineral/sulfides/pyrite/pyrite.htm">
<img src = "dodecahedron/pyrite.jpg">
% </a>
</div>

The Alhambra, a Moorish palace in Granada
built around 1300, has tile patterns with at least 13 of the 17 possible 
symmetry groups:

1) Branko Gr&uuml;nbaum, What symmetry groups are present in the 
Alhambra?, Notices of the AMS, 53 (2006),
670-673.  Also available at <a href = "http://www.ams.org/notices/200606/comm-grunbaum.pdf">http://www.ams.org/notices/200606/comm-grunbaum.pdf</a>

You can see some of these patterns here:

2) Moresque tiles, <a href = "http://www.spsu.edu/math/tile/grammar/moor.htm">http://www.spsu.edu/math/tile/grammar/moor.htm</a>

Recently, Peter Lu and Paul Steinhardt discovered that Islamic tile 
designs also include "quasicrystals".  A perfectly repetitive tiling
can't have 5-fold symmetry.  Nor can a 3-dimensional crystal: that's why
the dodecahedra formed by pyrite aren't regular.  But by using 
patterns that never quite repeat, the Islamic artists
achieved <i>approximate</i> 5-fold symmetry:

3) Peter J. Lu and Paul J. Steinhardt, Decagonal and quasi-crystalline
tilings in medieval Islamic architecture, Science 315 (2007),
1106-1110.  

Here's an example from the <a href = "http://archnet.org/library/sites/one-site.tcl?site_id=7544">I'timad al-Daula mausoleum</a> in the Indian city of
Agra, built by Islamic conquerors in 1622 - together with a more 
mathematical version constructed by Lu and Steinhardt:

<div align = "center">
<a href = "http://www.physics.harvard.edu/~plu/publications/Science_315_1106_2007_SOM.pdf">
<img width = "400" style = "border:none;" src = "quasicrystal_I'timid_al-Daula.jpg">
% </a>
</div>

Here's another, from the 
<a href = "http://archnet.org/library/sites/one-site.tcl?site_id=8380">Darb-i Imam shrine</a>
in Isfahan, Iran, also built 
in the 1600s:

<div align = "center">
<a href = "http://www.physics.harvard.edu/~plu/publications/Science_315_1106_2007_SOM.pdf">
<img width = "400" src = "quasicrystal_Darb-i_Imam.jpg">
% </a>
</div>

This came as a big surprise, since everyone had \emph{thought} that the 
math behind quasicrystals was first discovered by Penrose around 1974, 
then seen in nature by Shechtman, Blech, Gratias and Cahn in 1983.
It goes to show that the appeal of symmetry, even in its subtler forms,
is very old!  It also goes to show that you can make big discoveries
just by looking carefully at what's in front of you.


For more on quasicrystals, try this:

4) Steven Webber, Quasicrystals, <a href = "http://www.jcrystal.com/steffenweber/">http://www.jcrystal.com/steffenweber/</a>

Of course, the appeal of symmetry didn't end with ancient Greeks or
medieval Islamic monarchs.  It also seems to have gotten ahold of John
Fry, chief executive of Fry's Electronics - a chain of retail shops
whose motto is "Your best buys are always at Fry's".  In 1994 he 
set up something called the American Institute of Mathematics.  The
headquarters was in a Fry's store in Palo Alto - not very romantic.  
But last year, this institute announced plans to move to a full-scale 
replica of the Alhambra!

<div align = "center">
<a href = "http://aimath.org/about/morganhill.html">
<img "width = 500" src = "aim_alhambra.jpg">
% </a>
</div>

5) Associated Press, Silicon valley will get Alhambra-like castle,
August 18, 2006.  Available at <a href = "http://www.msnbc.msn.com/id/14412387/">http://www.jcrystal.com/steffenweber/</a>
 
And this week, the institute flexed its mighty PR muscles and coaxed 
reporters from the New York Times, BBC, Le Monde, Scientific American, 
Science News, and so on to write about a highly esoteric advance in 
our understanding of symmetry - a gargantuan calculation involving the 
Lie group E_{8}:

6) American Institute of Mathematics, Mathematicians map E_{8}, 
<a href = "http://aimath.org/E8">http://aimath.org/E8</a>

The calculation is indeed huge.  The \emph{answer} takes up 60 gigabytes of
data: the equivalent of 45 days of music in MP3 format.  If this 
information were written out on paper, it would cover Manhattan!

But what's the calculation \emph{about?} It almost seems a good
explanation of that would \emph{also} cover Manhattan.  I took a
stab at it here:

7) John Baez, News about E_{8}, 
<a href = "http://golem.ph.utexas.edu/category/2007/03/news_about_e8.html">http://golem.ph.utexas.edu/category/2007/03/news_about_e8.html</a>

but I only got as far as sketching a description of E_{8} and some gadgets
called R-polynomials.  Then come Kazhdan-Lusztig polynomials, and 
Kazhdan-Lusztig-Vogan polynomials....  For more details, follow
the links, especially to the page written by Jeffrey Adams, who led
the project.

In weeks to come, I'll say more about some topics tangentially related
to this calculation - especially flag varieties, representation theory
and the Weil conjectures.  I may even talk about Kazhdan-Lusztig polynomials!

For starters, though, let's just look at some pretty pictures by John Stembridge
that hint at the majesty of E_{8}.  Then I'll sketch the real subject 
of Weeks to come: symmetry, geometry, and "groupoidification".

To warm up to E_{8}, let's first take a look at D_{4}, D_{5}, E_{6}, and E_{7}.  

In "<A HREF = "week91.html">week91</A>" I spoke about the D_{4}
lattice.  To get this, first take a bunch of equal-sized spheres in 4
dimensions.  Stack them in a hypercubical pattern, so their centers
lie at the points with integer coordinates.  A bit surprisingly,
there's a lot of room left over - enough to fit in another copy of
this whole pattern: a bunch of spheres whose centers lie at the points
with \emph{half-integer} coordinates!

If you stick in these extra spheres, you get the densest known packing
of spheres in 4 dimensions.  Their centers form the "D_{4}
lattice".  It's an easy exercise to check that each sphere
touches 24 others.  The centers of these 24 are the vertices of a
marvelous shape called the "24-cell" - one of the six
4-dimensional Platonic solids.  It looks like this:

<div align = "center">
<a href = "octonions/conway_smith">
<img style = "border:none;" src = "octonions/conway_smith/24_cell.jpg">
% </a>
</div>

8) John Baez, picture of 24-cell, in a review of On Quaternions
and Octonions: Their Geometry, Arithmetic and Symmetry, by John H.
Conway and Derek A. Smith, available at 
<a href = "http://math.ucr.edu/home/baez/octonions/conway_smith/">
http://math.ucr.edu/home/baez/octonions/conway_smith/</a>

Here I'm using a severe form of perspective to project 4 dimensions down 
to 2.  The coordinate axes are drawn as dashed lines; the solid lines are 
the edges of the 24-cell.

How about in 5 dimensions?  Here the densest known packing of spheres
uses the "D_{5} lattice".  This is a lot like the D_{4}
lattice... but only if you think about it the right way.

Imagine a 4-dimensional checkerboard with "squares" - really
hypercubes! - alternately colored red and black.  Put a dot in the
middle of each black square.  Voila!  You get a rescaled version of
the D_{4} lattice.  It's not instantly obvious that this matches my
previous description, but it's true.

If you do the same thing with a 5-dimensional checkerboard, you get 
the "D_{5} lattice", by definition.  This gives the densest known 
packing of spheres in 5 dimensions.  In this packing, each sphere
has 40 nearest neighbors.  The centers of these nearest neighbors 
are the vertices of a solid that looks like this:

<div align = "center">
<a href = "http://www.math.lsa.umich.edu/~jrs/">
<img style = "border:none;" src = "d5_stembridge.jpg">
% </a>
</div>


9) John Stembridge, D_{5} root system, available at
<a href = "http://www.math.lsa.umich.edu/~jrs/data/coxplanes/">http://www.math.lsa.umich.edu/~jrs/data/coxplanes/</a>

If you do the same thing with a 6-dimensional checkerboard, you get
the "D_{6} lattice"... and so on.  

However, in 8 dimensions something cool happens.  If you pack spheres
in the pattern of the D_{8} lattice, there's enough room left
to stick in an extra copy of this whole pattern!  The result is called
the "E_{8} lattice".  It's twice as dense as the
D_{8} lattice.

If you then take a well-chosen 7-dimensional slice through the origin
of the E_{8} lattice, you get the E_{7} lattice.  And
if you take a well-chosen 6-dimensional slice of this, you get the
E_{6} lattice.  For precise details on what I mean by
"well-chosen", see "<A HREF =
"week65.html">week65</A>".

E_{6} and E_{7} give denser packings of spheres than
D_{6} and D_{7}.  In fact, they give the densest known packings
of spheres in 6 and 7 dimensions!

In the E_{6} lattice, each sphere has 72 nearest neighbors.  They form
the vertices of a solid that looks like this:

<div align = "center">
<a href = "http://www.math.lsa.umich.edu/~jrs/">
<img style = "border:none;" src = "e6_stembridge.jpg">
% </a>
</div>

10) John Stembridge, E_{6} root system, available at
<a href = "http://www.math.lsa.umich.edu/~jrs/data/coxplanes/">http://www.math.lsa.umich.edu/~jrs/data/coxplanes/</a>

In the E_{7} lattice, each sphere has 126 nearest neighbors.  They form
the vertices of a solid like this:

<div align = "center">
<a href = "http://www.math.lsa.umich.edu/~jrs/">
<img style = "border:none;" src = "e7_stembridge.jpg">
% </a>
</div>

11) John Stembridge, E_{7} root system, available at
<a href = "http://www.math.lsa.umich.edu/~jrs/data/coxplanes/">http://www.math.lsa.umich.edu/~jrs/data/coxplanes/</a>


In the E_{8} lattice, each sphere has 240 nearest neighbors.  They form
the vertices of a solid like this:

<div align = "center">
<a href = "http://www.math.lsa.umich.edu/~jrs/">
<img style = "border:none;" src = "e8_stembridge_small.jpg">
% </a>
</div>

12) John Stembridge, E_{8} root system, available at
<a href = "http://www.math.lsa.umich.edu/~jrs/data/coxplanes/">http://www.math.lsa.umich.edu/~jrs/data/coxplanes/</a>


Faithful readers will know I've discussed these lattices often before.
For how they give rise to Lie groups, see "<A HREF =
"week63.html">week63</A>".  For more about "ADE
classifications", see "<A HREF =
"week64.html">week64</A>" and "<A HREF =
"week230.html">week230</A>".  I haven't really added much this
time, except Stembridge's nice pictures.  I'm really just trying to
get you in the mood for a big adventure involving all these ideas: the
Tale of Groupoidification!

If we let this story lead us where it wants to go, we'll meet 
all sorts of famous and fascinating creatures, such as:

<ul>
<li>
 Coxeter groups, buildings, and the quantization of logic  
</li>
<li>
 Hecke algebras and Hecke operators 
</li>
<li>
 categorified quantum groups and Khovanov homology 
</li>
<li>
 Kleinian singularities and the McKay correspondence 
</li>
<li>
 quiver representations and Hall algebras 
</li>
<li>
 intersection cohomology, perverse sheaves and Kazhdan-Lusztig theory 
</li>
</ul>


% parser failed at source line 349
