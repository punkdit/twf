
% </A>
% </A>
% </A>
\week{December 14, 2005 }


This week I want to mention a couple of papers lying on the interface of
physics, topology, and higher-dimensional algebra.  But first, some
astronomy pictures... and a bit about the mathematical physicist Hamilton!

I like this photo of a jet emanating from the black hole in the 
center of galaxy M87:

<BR>
<DIV ALIGN = CENTER>
<A HREF = "http://hubblesite.org/newscenter/newsdesk/archive/releases/2005/12/image/o">
<IMG SRC = m87_jet.jpg>
% </A>
</DIV>
<BR>

1) NASA and John Biretta, M87,
<A HREF = "http://hubblesite.org/newscenter/newsdesk/archive/releases/2005/12/image/o">http://hubblesite.org/newscenter/newsdesk/archive/releases/2005/12/image/o</A>

M87 is a giant elliptical galaxy.  It's long been known as a powerful 
radio source, and now we know why: there's a supermassive black hole 
in the center, about 3 billion times the mass of our Sun.  As matter 
spirals into this huge black hole, it forms an "accretion disk", and
some gets so hot that it shoots out in a jet, as envisioned here:

<BR>
<DIV ALIGN = CENTER>
<A HREF = "http://maxim.gsfc.nasa.gov/docs/science/science.html">
<IMG SRC = accretion_disk.jpg>
% </A>
</DIV>
<BR>

2) NASA, MAXIM: Micro-Arcsecond X-ray Imaging Mission,
<A HREF = "http://maxim.gsfc.nasa.gov/docs/science/science.html">http://maxim.gsfc.nasa.gov/docs/science/science.html</A>

Accretion disks and jets are common at many different scales in our 
universe.  They're just nature's way of letting a bunch of matter fall
in under its own gravitation while losing angular momentum and energy.  
We see them when dust clouds collapse to form stars, we see them when 
black holes sucks in mass from companion stars, and they're probably 
also responsible for slow \gamma  ray bursts as huge stars collapse when 
they run out of fuel - see "<A HREF = "week204.html">week204</A>" for that story.  

But, among the biggest accretion disks and jets are those surrounding
supermassive black holes in the middle of galaxies.  These are probably
responsible for all the "active galactic nuclei" or "quasars" that we
see.  In the case of M87 the jet is enormous: 5000 light years long!  
To get a sense of the scale, look at the small white specks away from the 
jet in the next picture.  These are globular clusters: clusters containing 
between ten thousand and a million stars.  

<BR>
<DIV ALIGN = CENTER>
<A HREF = "http://antwrp.gsfc.nasa.gov/apod/ap000706.html">
<IMG SRC = m87jet_hst.jpg>
% </A>
</DIV>
<BR>

3) A jet from galaxy M87, Astronomy Picture of the Day, July 6, 2000,
<A HREF = "http://antwrp.gsfc.nasa.gov/apod/ap000706.html">http://antwrp.gsfc.nasa.gov/apod/ap000706.html</A>

The jet in M87 is so hot that it emits not just radio waves and visible 
light, but even X-rays, as seen by the Chandra X-ray telescope:

<BR>
<DIV ALIGN = CENTER>
<A HREF = "http://chandra.harvard.edu/photo/2001/0134/">
<IMG SRC = "m87_xray_radio_optical.jpg">
% </A>
</DIV>
<BR>

4) M87: Chandra sheds light on the knotty problem of the M87 jet,
<A HREF = "http://chandra.harvard.edu/photo/2001/0134/">
http://chandra.harvard.edu/photo/2001/0134/</A>

It seems the jet consists mainly of electrons moving at relativistic
speeds, focused by the magnetic field of the accretion disk.  They 
come in blobs called "knots".  People can actually see these blobs 
moving out, getting brighter and dimmer.

In fact, many galaxies have super-massive black holes at their centers
with jets like this one.  The special thing about M87 is that it's fairly 
nearby, hence easy to see.  M87 is the biggest galaxy in the Virgo Cluster. 
This is the closest galaxy cluster to us, about 50 million light years away.
That sounds pretty far, but it's only 1000 times the radius of the Milky
Way.  If the Milky Way were a pebble, M87 would be only a stone's throw
away.  So, even amateur astronomers - really good ones, at least - can take 
photos of M87 that show the jet.  But here's a high-quality picture produced
by Robert Lupton using data from the Sloan Digital Sky Survey - you can see 
the jet in light blue:

<BR>
<DIV ALIGN = CENTER>
<A HREF = "http://www.astro.princeton.edu/~rhl/PrettyPictures/">
<IMG HEIGHT = 500 WIDTH = 500 SRC = "m87_core.jpg">
% </A>
</DIV>
<BR>

5) Robert Lupton and the Sloan Digital Sky Survey Consortium, The
central regions of M87, <A HREF =
"http://www.astro.princeton.edu/~rhl/PrettyPictures/">http://www.astro.princeton.edu/~rhl/PrettyPictures/</A>

Backing off a bit further, let's take a look at the Virgo Cluster.  It
contains over a thousand galaxies, but we can tell it's fairly new as
clusters go, since it consists of a bunch of "subclusters"
that haven't merged yet.  Our galaxy, and indeed the whole Local Group
to which it belongs, is being pulled towards the Virgo Cluster and
will eventually join it.  Here's a nice closeup of part of the Virgo
Cluster:

<BR>
<DIV ALIGN = CENTER>
<A HREF = "http://burro.astr.cwru.edu/Schmidt/Virgo/">
<IMG HEIGHT = 300 WIDTH = 500 SRC = "virgo_cluster.jpg">
% </A>
</DIV>
<BR>

6) Chris Mihos, Paul Harding, John Feldmeier and Heather Morrison,
Deep imaging of the Virgo Cluster, <A HREF = "http://burro.astr.cwru.edu/Schmidt/Virgo/">http://burro.astr.cwru.edu/Schmidt/Virgo/</A>

Finally, just for fun, something unrelated - and more mysterious.  It's 
called "Hoag's object":

<BR>
<DIV ALIGN = CENTER>
<A HREF = "http://heritage.stsci.edu/2002/21/">
<IMG HEIGHT = 500 WIDTH = 500 SRC = "hoag.jpg">
% </A>
</DIV>
<BR>

7) The Hubble Heritage Project, Hoag's Object, 
<A HREF = "http://heritage.stsci.edu/2002/21/">http://heritage.stsci.edu/2002/21/</A>

It's a ring-shaped galaxy full of hot young blue stars surrounding a ball 
of yellower stars.  Nobody knows how it formed: perhaps by a collision
of two galaxies?  Such collisions are fairly common, but they don't 
typically create this sort of structure.  

The weirdest part is that
inside the ring, in the upper right, you can see \emph{another} ring galaxy 
in the distance! 
Maybe an advanced civilization over there enjoys this form of art?
Probably not, but if it turns out to be true, you heard it here first.

Anyway... back here on Earth, in the summer of 2004, I visited Dublin for a 
conference on general relativity called GR17.  As recounted in "<A HREF = "week207.html">week207</A>", 
this was where Hawking admitted defeat in his famous bet with John Preskill 
about information loss due to black hole evaporation.  In August of this 
year, Hawking finally came out with a short paper on the subject:

8) Stephen W. Hawking, Information loss in black holes, available as
<A HREF = "http://xxx.lanl.gov/abs/hep-th/0507171">hep-th/0507171</A>.

I spent a lot of time talking to physicists, but I also wandered around 
Dublin a bit.  Besides listening to some great music at a pub called 
Cobblestones - Kevin Rowsome plays a mean uilleann pipe! - and tracking 
down some sites mentioned in James Joyce's novel "Ulysses", I went with 
Tevian Dray on a pilgrimage to Brougham Bridge.  

Tevian Dray is an expert on the octonions, and Brougham Bridge is where 
Hamilton carved his famous formula defining the quaternions!  Now there 
is a plaque commemorating this event, which reads:

<DIV ALIGN = CENTER>
                    Here as he walked by <br>
                 on the 16th of October 1843 <br>
                 Sir William Rowan Hamilton <br>
               in a flash of genius discovered <br>
                the fundamental formula for <br>
                 quaternion multiplication <br>
                i^{2} = j^{2} = k^{2} = ijk = -1 <br>
             \text{\&}  cut it on a stone of this bridge
</DIV>

It does't mention that Hamilton had been racking his brain for the
entire month of October trying to solve this problem: "flashes of
genius" favor the prepared mind.  But it's a nice story and a nice place.
My friend Tevian Dray took some photos, which you can see here:

9) John Baez, Dublin, <A HREF = "http://math.ucr.edu/home/baez/dublin/">http://math.ucr.edu/home/baez/dublin/</A>

It was a bit of a challenge finding Brougham Bridge, since nobody at the main
bus station gave us correct information about which bus went there - except 
the bus driver who finally took us there.  So, to ease your way in case
you want to make your own pilgrimage, the above page includes directions. 
And now, thanks to Dirk Schlimm, it also includes a link to a map showing 
the bridge!  

Speaking of Hamilton, Theron Stanford recently sent me an answer to one of 
life's persistent questions: why is momentum denoted by the letter p?  

Since momentum and position play fundamental roles in Hamiltonian mechanics, 
and they're denoted by p and q, one wonders: could this notation be related 
to Hamilton's alcoholism in later life?  After all, some claim the saying 
\emph{mind your p's and q's} began as a friendly Irish warning not
to imbibe too many pints and quarts!  So, maybe he used these letters
in his work on physics as a secret plea for help.

Umm... probably not.  Just kidding.  But in the absence of hard facts, 
speculation runs rampant.  So, I'm glad Stanford provided some of the former, 
to squelch the latter.

He sent me this email:



% parser failed at source line 288
