
% </A>
% </A>
% </A>
\week{June 12, 2000 }

Elliptic cohomology sits at the intersection of several well-travelled
mathematical roads.  It boasts fascinating connections with homotopy
theory, string theory, elliptic curves, modular forms, and the mysterious 
ubiquity of the number 24.  This makes it very fascinating, but also a bit 
intimidating to anyone who is not already an expert on all these subjects.  

Is \emph{anyone} 
actually an expert on all these subjects?  Perhaps Graeme Segal 
is!  After all, he became famous for his work on homotopy theory, he 
\emph{invented} the axioms of conformal field theory - borrowing lots of 
ideas from string theory, of course - and I'm sure he mastered the 
theory of elliptic curves one weekend when he was a kid.  So to learn 
about elliptic cohomology, one should really start here:

1) Graeme Segal, Elliptic cohomology, Asterisque 161-162 (1988), 187-201.  

Another good reference is this proceedings of a conference held at 
Princeton in 1986:

2) Peter S. Landweber, editor, Elliptic Curves and Modular Forms in
Algebraic Topology, Springer-Verlag Lecture Notes in Mathematics 1326,
Springer, Berlin, 1988.

This book is also helpful:

3) Charles B. Thomas, Elliptic Cohomology, Kluwer, Dordrecht, 1999.   

though I'm afraid it's a bit long on details and short on the big 
picture and physics intuition.  For that, you might have to try this:

4) Edward Witten, Elliptic genera and quantum field theory, Comm.
Math. Phys. 109 (1987), 525-536.

Also try this book, if you can get ahold of it:

5) Friedrich Hirzebruch, Thomas Berger and Rainer Jung, Manifolds 
and modular forms, translated by Peter S. Landweber, Vieweg, 
Braunschweig (a publication of the Max Planck Institute for 
Mathematics in Bonn), 1992.

Now to have a snowball's chance in hell of understanding elliptic
cohomology, you need to understand complex oriented cohomology theories.
So I have to start by telling you what \emph{those} are.  This will be sort
of a crash course in algebraic topology.  By the time I'm done with that,
I'll probably be too worn out to talk about elliptic cohomology - but
at least I'll have laid the groundwork.

So: what's a "generalized cohomology theory"?

This is a gadget that eats a topological space X and spits out a sequence 
of abelian groups h^{n}(X).   To be a generalized cohomology theory, this
gadget must satisfy a bunch of axioms called the Eilenberg-Steenrod
axioms.  The most basic example is so-called ordinary cohomology, so
when you're first learning this stuff the main motivation for the
Eilenberg-Steenrod axioms is that they're all satisfied by ordinary
cohomology.   But there are lots of other examples: various flavors of
K-theory, cobordism theory, and so on.  Eventually, you learn that
underlying any generalized cohomology theory there is a list of spaces
E(n) such that

                        h^{n}(X) = [X, E(n)]

where the right-hand side is the set of homotopy classes of maps from X
to E(n).  We say this list of spaces E(n) "represents" the generalized
cohomology theory.  Moreover, these spaces fit together to form a 
"spectrum", meaning that the space of based loops in E(n) is E(n-1).  
It follows that each space E(n) is an infinite loop space: a space of
loops in a space of loops in a space of loops in... where you can go
on as far as you like.  

Conversely, given an infinite loop space E(0), we can use it to cook up 
a spectrum and thus a generalized cohomology theory.  So generalized
cohomology theories, spectra and infinite loop spaces are almost the
same thing.  

But what's so important about them?

Well, secretly an infinite loop space is nothing but a homotopy
theorist's version of an abelian group.  A bit more technically, we
could call it a "homotopy coherent abelian group".  By this I mean a
space with a continuous binary operation satisfying all the usual laws
for an abelian group \emph{up to homotopy}, where these homotopies satisfy
all the nice laws you can imagine \emph{up to homotopy}, and so on ad
infinitum.   In the context of homotopy theory, this is almost as good 
as an abelian group.  Pretty much anything a normal mathematician can do
with an abelian group, a homotopy theorist can do with an infinite loop
space!

For example, normal mathematicians often like to take an abelian group 
and equip it with an extra operation called "multiplication" that
makes it into a \emph{ring}.  Homotopy theorists like to do the same
for infinite loop spaces.  But of course, the homotopy theorists
only demand that the ring axioms hold \emph{up to homotopy}, where the
homotopies satisfy a bunch of nice laws \emph{up to homotopy}, and so on. 
Usually they do this in the context of spectra rather than infinite loop
spaces - a distinction too technical for me to worry about here! - so
they call this sort of thing a "ring spectrum".   Similarly, corresponding
to a commutative ring, the homotopy theorists have a notion called an
"E_{\infty } ring spectrum".  The word 
"E_{\infty }" is just a funny way
of saying that the commutative law holds up to homotopy, with the 
homotopies satisfying a bunch of laws up to homotopy, etcetera. 

If you start with a ring spectrum, the corresponding cohomology theory
will have products.  In other words, the cohomology groups h^{n}(X)
of any space X will fit together to form a graded ring called h^{*}(X)
- the star stands for a little blank where you can stick in any number 
"n".  
And if your ring spectrum is an E_{\infty } ring spectrum, h^{*}(X) 
will be graded-commutative.  This is what happens in most of really famous 
generalized cohomology theories.  For example, the ordinary cohomology 
of a space is actually a graded-commutative ring with a product called
the "cap product", and similar things are true for the most popular 
flavors of K-theory and cobordism theory.

Of course, it's quite a bit of work to make all this stuff precise:
people spent a lot of energy on it back in the 1970's.  But it's very
beautiful, so everybody should learn it.  For the details, try:

6) J. Adams, Infinite Loop Spaces, Princeton U. Press, Princeton, 1978.

7) J. Adams, Stable Homotopy and Generalized Homology, Chicago Lectures
in Mathematics, U. Chicago Press, Chicago, 1974.

8) J. P. May, The Geometry of Iterated Loop Spaces, Lecture Notes in
Mathematics 271, Springer Verlag, Berlin, 1972.

9) J. P. May, F. Quinn, N. Ray and J. Tornehave, E_{\infty } Ring Spaces
and E_{\infty } Ring Spectra, Lecture Notes in Mathematics 577, Springer
Verlag, Berlin, 1977.

10) G. Carlsson and R. Milgram, Stable homotopy and iterated loop spaces,
in Handbook of Algebraic Topology, ed. I. M. James, North-Holland, 1995.

Now, there's a particularly nice class of generalized cohomology
theories called "complex oriented cohomology theories".   Elliptic
cohomology is one of these, so to understand elliptic cohomology you
first have to study these guys a bit.  Instead of just giving you 
the definition, I'll lead up to it rather gradually....

Let's start with the integers, Z.  These form an abelian group under
addition, so by what I said above they are a pitifully simple special
case of an infinite loop space.  So there's some space with a basepoint 
called K(Z,1) such that the space of all based loops in K(Z,1) is Z.  

Be careful here: 
I'm now using the word "is" the way homotopy theorists 
do!  I really mean the space of based loops in K(Z,1) is <em>homotopy 
equivalent</em> to Z.  But since we're doing homotopy theory, that's good 
enough.   

Okay: so there's a space K(Z,1) such that the space of all based loops
in K(Z,1) is Z.  Similarly, there's a space K(Z,2) such that the space 
of all based loops in K(Z,2) is K(Z,1).  And so on... that's what it
means to say that Z is an infinite loop space.

These spaces K(Z,n) are called "Eilenberg-MacLane spaces", and they fit
together to form a spectrum called the Eilenberg-MacLane spectrum. Since
it's built using only the integers, this is the simplest, nicest
spectrum in the world.  Thus the generalized cohomology theory it
represents has got to be something simple and nice.  And it is: it's
just ordinary cohomology!  

But what do the spaces K(Z,n) actually look like?  

Well, for starters, K(Z,0) is just Z, by definition.

K(Z,1) is just the circle, S^{1}.   
You can check that the space of based
loops in S^{1} is homotopy equivalent to Z - the key is that such loops
are classified up to homotopy by an integer called the \emph{winding number}.
In quantum physics, K(Z,1) usually goes by the name U(1) - the group
of unit complex numbers, or "phases".   

K(Z,2) is a bit more complicated: it's infinite-dimensional complex
projective space, CP^{\infty }!  I talked a bunch about projective 
spaces in "<A HREF = "week106.html">week106</A>".  There I only talked about finite-dimensional ones
like CP^{n}, but you can define CP^{\infty } 
as a "direct limit" of these 
as n approaches \infty , using the fact that CP^{n} sits inside 
CP</sup>n+1</sup> as
a subspace.   Alternatively, you can take your favorite complex Hilbert
space H with countably infinite dimension and form the space of all 
1-dimensional subspaces in H.  This gives a slightly fatter version of
CP^{\infty }, but it's homotopy equivalent, and it's a very natural 
thing to study if you're a physicist: it's just the space of all "pure 
states" of the quantum system whose Hilbert space is H.  

How about K(Z,3)?  Well, I don't know a nice geometrical description of
this one.  And this really pisses me off!  There should be some nice way
to think of K(Z,3) as some sort of infinite-dimensional manifold.  What
is it?  Does anyone know?  Jean-Luc Brylinski raised this question at the
Conference on Higher Category Theory and Physics in 1997, and it's been
bugging me ever since.  From the work of Brylinski which I summarized
in "<A HREF = "week25.html">week25</A>", it's clear that a good answer should shed light on stuff like
quantum theory and string theory.  Basically, the point is that the integers, 
the group U(1), and infinite-dimensional complex projective space are all 
really important in quantum theory.  This is perhaps more obvious for the
latter two spaces - the integers are so basic that it's hard to see what's
so "quantum-mechanical" about them.   However, since each of these spaces 
is just the loop space of the next, they're all part of tightly linked 
sequence... and I want to know what comes next!  

But I'm digressing.  I really want to focus on K(Z,2), or in other
words, infinite-dimensional complex projective space.  Note that there's
an obvious complex line bundle over this space.  Remember, each point in
CP^{\infty } is 
really a 1-dimensional subspace in some Hilbert space H. 
So we can use these 1-dimensional subspaces as the fibers of a complex
line bundle over CP^{\infty }, 
called the "canonical bundle".  I'll
call this line bundle L.

The complex line bundle L is important because it's "universal": all 
the rest can be obtained from this one!  More precisely, suppose we
have any topological space X and any map 

f: X \to  CP^{\infty }  

Then we can form a complex line bundle over X whose fiber over any point
x is just the fiber of L over the point f(x).  This bundle is called the
"pullback" of L by the map f.   And the really cool part 
is that \emph{any} 
complex line bundle over \emph{any} space X is isomorphic to the pullback of 
L by some map!  Even better, two such line bundles are isomorphic if and 
only if the maps f defining them are homotopic!  This reduces the study 
of many questions about complex line bundles to the study of this guy L.

For example, suppose we want to classify complex line bundles over any 
space X.  From what I just said, this task is equivalent to the task of
classifying homotopy classes of maps 

f: X \to  CP^{\infty }.

But remember, CP^{\infty } is the Eilenberg-Maclane space K(Z,2), and
the Eilenberg-Maclane spectrum represents ordinary cohomology!  So

[X, CP^{\infty }] = [X, K(Z,2)] = H^{2}(X)

where H^{2}(X) 
stands for the 2nd ordinary cohomology group of X.   So 
the following things are really the same:

<UL>
<LI> isomorphism classes of complex line bundles over X
<LI> homotopy classes of maps from X to CP^{\infty }
<LI> elements of the ordinary cohomology group H^{2}(X).
</UL>

This is great, because it gives us three different viewpoints to play
with.  In particular, H^{2}(X) 
is easy to compute - anyone who has taken 
a basic course on algebraic topology can do it.  But the other two
viewpoints are nice and geometrical, so they let us do things with H^{2}(X)
that we might not have thought of otherwise.

So now you know this: if you hand me a complex line bundle over X, I can 
cook up an element of H^{2}(X).   People call this the "first Chern class"
of the line bundle.   If you hand me two complex line bundles, I can tell 
if they're isomorphic by seeing if their first Chern classes are equal.  
Conversely, if you hand me any element of H^{2}(X), I can cook up a complex 
line bundle over X whose first Chern class is that element.

Of course, I haven't really explained \emph{how} I cook up all these things.
To learn that, you need to study this stuff a bit more.

But let's consider a couple of examples.  Suppose X is the 2-sphere S^{2}.
Since

H^{2}(S^{2}) = Z

this means that first Chern class of a line bundle over S^{2} is secretly
just an integer.  People call this the "first Chern number" of the line
bundle.  The first physicist to get excited about this was Dirac, who 
bumped into this idea when thinking about magnetic monopoles and charge
quantization.  Dirac didn't know about complex line bundles and Chern 
classes - he was just studying the change of phase of an electrically
charged particle as you move it around in the magnetic field produced by
a monopole!  But later, the physicist Yang met the mathematician Chern
and translated Dirac's work into the language of line bundles.  See

11) C. N. Yang, Magnetic monopoles, fiber bundles and gauge field,
in Selected Papers, 1945--1980, with Commentary, W. H. Freeman and 
Company, San Francisco, 1983.  

for the full story.

Next let's try a curiously self-referential example.  It should be fun 
to classify complex line bundles on CP^{\infty }, since this is where 
the universal one lives!  So let's take X = CP^{\infty }.  Since 
CP^{\infty } is K(Z,2), a little abstract nonsense shows that it's
ordinary 2nd cohomology group is Z:

H^{2}(CP^{\infty }) = [CP^{\infty }, CP^{\infty }] = Z.

This means that the first Chern class of a complex line bundle over 
CP^{\infty } is secretly just an integer.  But what's the first Chern 
class of the universal complex line bundle, L?  Well, this bundle is 
the pullback of itself via the \emph{identity} map

1: CP^{\infty } \to  CP^{\infty }

and this map corresponds to the element 1 in [CP^{\infty }, CP^{\infty }]
= Z. So the first Chern class of L is 1.  See how tautologous this 
argument is?   It sounds like it's saying something profound, but once
you understand it, it's really just saying 1 = 1.

The first Chern class of the universal bundle L is really important, 
so let's call it c.  It's important because it's universal: it gives 
us a nice way to think of the first Chern class of \emph{any} complex line 
bundle.    Up to isomorphism, any complex line bundle over any space 
X comes from some map 

f: X \to  CP^{\infty }

so to compute the first Chern class of this line bundle, we can just
work out f^{*}(c), where

f^{*}: H^{2}(CP^{\infty }) \to  H^{2}(X)

is the map induced by f.  If you don't see why this is true, think about 
it a while - it's just a big fat tautology!   

The ideas we've been discussing raise some obvious questions.  For
example, H^{2}(X) isn't just a set: it's an abelian group.   We already
knew this from our basic course in algebraic topology, and now we also
know another explanation: CP^{\infty } is an infinite loop space, 
so it's like an abelian group for the purposes of homotopy theory.  In 
fact, this particular infinite loop space actually \emph{is} 
an abelian group.  
Maps from anything into an abelian group form an abelian group, which makes 

H^{2}(X) = [X, CP^{\infty }] 

into an abelian group.    But now you're dying to know: what exactly do 
the product map

m: CP^{\infty } x CP^{\infty } \to  CP^{\infty } 

and the inverse map

i: CP^{\infty } \to  CP^{\infty } 

look like?  And what does all this mean for the set of isomorphism 
classes of complex line bundles on X?  It's an abelian group - but 
what are products and inverses like in this abelian group?

Well, I won't answer the first question here: there's a very nice 
explicit answer, and you can describe it in terms of particles and 
antiparticles running around on the Riemann sphere, but it would be 
too much of a digression to talk about it here.  To learn more, 
study the "Thom-Dold theorem" and also some stuff about "configuration 
spaces" in topology:

12) Dusa McDuff, Configuration spaces of positive and negative particles,
Topology 14 (1975), 91-107.

The second question is much easier: the set of isomorphism classes of 
complex line bundles on a space X becomes an abelian group with <em>tensor


% parser failed at source line 459
