
% </A>
% </A>
% </A>
\week{May 6, 1998 }

Now that I'm hanging out with the gravity crowd at Penn State, I might
as well describe what's been going on here lately.

First of all, Ashtekar and Krasnov have written an expository account
of their work on the entropy of quantum black holes:

1) Abhay Ashtekar and Kirill Krasnov, Quantum geometry and black holes, 
preprint available as 
<A HREF = "http://xxx.lanl.gov/abs/gr-qc/9804039">gr-qc/9804039</A>.

But if you prefer to see a picture of a quantum black hole
without any equations, try:

2) <A HREF = "http://shiva.nirvana.phys.psu.edu/~krasnov/research.html">Kirill
Krasnov's research webpage</A>.

You'll see a bunch of spin networks poking the horizon, giving it
area and curvature.  Of course, this is just a theory.  

Second, there's been a burst of new work studying quantum gravity in
terms of spin foams.  A spin foam looks a bit like a bunch of soap
suds - with the faces of the bubbles and the edges where the bubbles
meet labelled by spins j = 0, 1/2, 1, 3/2, etc..  Spin foams are an
attempt at a quantum description of the geometry of
spacetime.  If you slice a spin foam with a hyperplane representing 
"t = 0" you get a spin network: a graph with its edges and vertices
labelled by spins.  Spin networks have been used in quantum gravity
for a while now to describe the geometry of space at a given time, so
it's natural to hope that they're a slice of something that describes
the geometry of spacetime.

As usual in quantum gravity, it's too early to tell if this approach
will work.  As usual, it has lots of serious problems.  But before
going into the problems, let me remind you how spin foams are supposed
to work.

To relate spin foams to more traditional ideas about spacetime, one
can consider spin foams living in a triangulated 4-manifold: one spin
foam vertex sitting in each 4-simplex, one spin foam edge poking
through each tetrahedron, and one spin foam face intersecting each
triangle.  Labelling the spin foam edges and faces with spins is
supposed to endow the triangulated 4-manifold with a "quantum
4-geometry".  In other words, it should let us compute things like the
areas of the triangles, the volumes of the tetrahedra, and the
4-volumes of the 4-simplices.  There are some arguments going on now
about the right way to do this, but it's far from an arbitrary
business: the interplay between group representation theory and
geometry says a lot about how it should go.  In the simplified case of
3-dimensional spacetime, it's fairly well understood - the hard
part, and the fun part, is getting it to work in 4 dimensions.

Assuming we can do this, the next trick is to compute an amplitude
for each spin foam vertex in a nice way, much as one computes
amplitudes for vertices of Feynman diagrams.  A spin foam vertex is
supposed to represent an "event" - if we slice the spin
foam by a hyperplane we get a spin network, and as we slide this slice
"forwards in time", the spin network changes its topology
whenever we pass a spin foam vertex.  The amplitude for a vertex tells
us how likely it is for this event to happen.  As usual in quantum
theory, we need to take the absolute value of an amplitude and square
it to get a probability.

We also need to compute amplitudes for spin foam edges and faces,
called "propagators", in analogy to the amplitudes one computes for
the edges of Feynman diagrams.  Multiplying all the vertex amplitudes
and propagators for a given spin foam, one gets the amplitude for the
whole spin foam.  This tells us how likely it is for the whole spin
foam to happen.

Barrett and Crane came up with a specific way to do all this stuff,
Reisenberger came up with a different way, I came up with a general
formalism for understanding this stuff, and now people are busy
arguing about the merits of different approaches.  Here are some
papers on the subject - I'll pick up where I left off in "<A
HREF = "week113.html">week113</A>".

3) Louis Crane, David N. Yetter, On the classical limit of the
balanced state sum, preprint available as <A HREF =
"http://xxx.lanl.gov/abs/gr-qc/9712087">gr-qc/9712087</A>.

The goal here is to show that in the limit of large spins, the
amplitude given by Barrett and Crane's formula approaches 

exp(iS)

where S is the action for classical general relativity - suitably
discretized, and in signature ++++.  The key trick is to use an idea
invented by Regge in 1961.

Regge came up with a discrete analog of the usual formula for the
action in classical general relativity.  His formula applies to a
triangulated 4-manifold whose edges have specified lengths.  In this
situation, each triangle has an "angle deficit" associated to it.
It's easier to visualize this two dimensions down, where each vertex
in a triangulated 2-manifold has an angle deficit given by adding up
angles for all the triangles having it as a corner, and then
subtracting 2 \pi .  No angle deficit means no curvature: the triangles
sit flat in a plane.  The idea works similarly in 4 dimensions.
Here's Regge's formula for the action: take each triangle in your
triangulated 4-manifold, take its area, multiply it by its angle
deficit, and then sum over all the triangles.  

Simple, huh?  In the continuum limit, Regge's action approaches the
integral of the Ricci scalar curvature - the usual action in general
relativity.  For more see:

4) T. Regge, General relativity without coordinates, Nuovo Cimento 19
(1961), 558-571.

So, Crane and Yetter try to show that in the limit of large spins, the
Barrett-Crane spin foam amplitude approaches exp(iS) where S is the
Regge action.  There argument is interesting but rather sketchy.
Someone should try to fill in the details!

However, it's not clear to me that the large spin limit is physically
revelant.  If spacetime is really made of lots of 4-simplices labelled
by spins, the 4-simplices have got to be quite small, so the spins
labelling them should be fairly small.  It seems to me that the right
limit to study is the limit where you triangulate your 4-manifold with
a huge number of 4-simplices labelled by fairly small spins.  After
all, in the spin network picture of the quantum black hole, it seems
that spin network edges labelled by spin 1/2 contribute most of the
states (see "<A HREF = "week112.html">week112</A>").  

When you take a spin foam living in a triangulated 4-manifold and
slice it in a way that's compatible with the triangulation, the spin
network you get is a 4-valent graph.  Thus it's not surprising that
Barrett and Crane's formula for vertex amplitudes is related to an
invariant of 4-valent graphs with edges labelled by spins.  There's
already a branch of math relating such invariants to representations
of groups and quantum groups, and their formula fits right in.  Yetter
has figured out how to generalize this graph invariant to n-valent
graphs with edges labelled by spins, and he's also studied more
carefully what happens when one "q-deforms" the whole business -
replacing the group by the corresponding quantum group.  This should
be related to quantum gravity with nonzero cosmological constant, if
all the mathematical clues aren't lying to us.  See:

5) David N. Yetter, Generalized Barrett-Crane vertices and invariants of 
embedded graphs, preprint available as <A HREF = "http://xxx.lanl.gov/abs/math.QA/9801131">math.QA/9801131</A>.

Barrett has also given a nice formula in terms of integrals for
the invariant of 4-valent graphs labelled by spins.  This is motivated
by the physics and illuminates it nicely:

6) John W. Barrett, The classical evaluation of relativistic spin
networks, preprint available as <A HREF = "http://xxx.lanl.gov/abs/math.QA/9803063">math.QA/9803063</A>.


Let me quote the abstract:

\begin{quote}

     The evaluation of a relativistic spin network for the classical
     case of the Lie group SU(2) is given by an integral formula over
     copies of SU(2).  For the graph determined by a 4-simplex this
     gives the evaluation as an integral over a space of geometries
     for a 4-simplex.

\end{quote}
    

Okay, so much for the good news.  What about the bad news?  To explain
this I need to get a bit more specific about Barrett
and Crane's approach.  

Their approach is based on a certain way to describe the geometry of a
4-simplex.  Instead of specifying lengths of edges as in the old Regge
approach, we specify bivectors for all its faces.  Geometrically, a
bivector is just an "oriented area element"; technically, the space of
bivectors is the dual of the space of 2-forms.  If we have a 4-simplex
in R^{4} and we choose orientations for its triangular faces, there's an
obvious way to associate a bivector to each face.  We get 10 bivectors
this way.

What constraints do these 10 bivectors satisfy?  They can't be
arbitrary!  First, for any four triangles that are all the faces of
the same tetrahedron, the corresponding bivectors must sum to zero.
Second, every bivector must be "simple" - it must be the wedge
product of two vectors.  Third, whenever two triangles are the faces
of the same tetrahedron, the sum of the corresponding bivectors must
be simple.

It turns out that these constraints are almost but \emph{not quite enough}
to imply that 10 bivectors come from a 4-simplex.  Generically, it
there are four possibilities: our bivectors come from a 4-simplex, 
the \emph{negatives} of our bivectors come from a 4-simplex, their <em>Hodge 


% parser failed at source line 242
