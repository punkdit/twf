
% </A>
% </A>
% </A>
\week{October 6, 1996 }


For a while now I've been meaning to finish talking about monads
and adjunctions, and explain what that has to do with the 4-color 
theorem.  But first I want to say a little bit more about "triality", 
which was the subject of "<A HREF = "week90.html">week90</A>".

Triality is a cool symmetry of the infinitesimal rotations in
8-dimensional space.  It was only last night, however, that I figured
out what triality has to do with 3 dimensions.  Since it's all about
the number \emph{three} obviously triality should originate in the
symmetries of \emph{three}-dimensional space, right?  Well, maybe
it's not so obvious, but it does.  Here's how.

Take good old three-dimensional Euclidean space with its usual
basis of unit vectors i, j, and k.  Look at the group of all
permutations of {i,j,k}.  This is a little 6-element group
which people usually call S_{3}, 
the "symmetric group on 3 letters".

Every permutation of {i,j,k} defines a linear transformation
of three-dimensional Euclidean space in an obvious way.  For 
example the permutation p with


\begin{verbatim}

p(i) = j,  p(j) = k,  p(k) = i
\end{verbatim}
    
determines a linear transformation, which we'll also call p, with


\begin{verbatim}

p(ai+ bj + ck) = aj + bk + ci.
\end{verbatim}
    
In general, the linear transformations we get this way
either preserve the cross product, or switch its sign.
If p is an even permutation we'll get


\begin{verbatim}

p(v) x p(w) = p(v x w)
\end{verbatim}
    
while if p is odd we'll get 


\begin{verbatim}

p(v) x p(w) = -p(v x w) = p(w x v).
\end{verbatim}
    
That's where triality comes from.  But now let's see what it
has to do with \emph{four}-dimensional space.  
We can describe four-dimensional
space using the quaternions.  A typical quaternion is something like


\begin{verbatim}

a + bi + cj + dk
\end{verbatim}
    
where a,b,c,d are real numbers, and you multiply quaternions
by using the usual rules together with the rules


$$

i^{2} = j^{2} = k^{2} = -1, 
ij =  k, jk =  i, ki =  j,
ji = -k, kj = -i, ik = -j
$$
    
Now, any permutation p of {i,j,k} also determines a linear 
transformation of the quaternions, which we'll also call p.  
For example, the permutation p I gave above has


\begin{verbatim}

p(a + bi + cj + dk) = a + bj + ck + di.
\end{verbatim}
    
The quaternion product is related to the vector cross product,
and so one can check that for any quaternions q and q' we
get 


\begin{verbatim}

p(qq') = p(q)p(q')
\end{verbatim}
    
if p is even, and


\begin{verbatim}

p(q'q) = p(q')p(q)
\end{verbatim}
    
if p is odd.  So we are getting triality to act as some sort
of symmetries of the quaternions.  

Now sitting inside the quaternions there is a nice lattice 
called the "Hurwitz integral quaternions".  It consists of 
the quaternions a + bi + cj + dk for which either a,b,c,d are 
all integers, or all half-integers.  Here I'm using physics jargon, 
and referring to any number that's an integer plus 1/2 as a 
"half-integer".  A half-integer is \emph{not} 
any number that's half 
an integer!   

You can think of this lattice as the 4-dimensional
version of all the black squares on a checkerboard.  
One neat thing is that if you multiply any two guys in
this lattice you get another guy in this lattice, so we
have a "subring" of the quaternions.  Another neat thing is 
that if you apply any permutation of {i,j,k} to a guy in this 
lattice, you get another guy in this lattice --- this is easy
to see.   So we are getting triality to act as some sort
of symmetries of this lattice.  And \emph{that} is what people
\emph{usually} call triality.  

Let me explain, but now let me use a lot of jargon.  (Having
shown it's all very simple, I now want to relate it to the
complicated stuff people usually talk about.  Skip this if
you don't like jargon.)  We saw how to get S_{3} to act as 
automorphisms and antiautomorphisms of R^{3} with its usual 
vector cross product... or alternatively, as automorphisms
and antiautomorphisms of the Lie algebra so(3).  From that
we got an action as automorphisms and antiautomorphisms of
the quaternions and the Hurwitz integral quaternions.  But
the Hurwitz integral quaternions are just a differently 
coordinatized version of the 4-dimensional lattice D_{4}!
So we have gotten triality to act as symmetries of the D_{4}
lattice, and hence as automorphisms of the Lie algebra D_{4},
or in other words so(8), the Lie algebra of infinitesimal 
rotations in 8 dimensions.  (For more on the D_{4} lattice see
"<A HREF = "week65.html">week65</A>", where I describe it using different, more traditional
coordinates.)

Actually I didn't invent all this stuff, I sort of dug
it out of the literature, in particular:

1)
John H. Conway and Neil J. A. Sloane, Sphere Packings, Lattices and
Groups, second edition, Grundlehren der mathematischen Wissenschaften
290, Springer-Verlag, 1993.
and

2) Frank D. (Tony) Smith, Sets and C^{n}; quivers and A-D-E; triality; 
generalized supersymmetry; and D4-D5-E6, preprint available as
<A HREF = "http://xxx.lanl.gov/abs/hep-th/9306011">hep-th/9306011</A>.

But I've never quite seen anyone come right out and admit
that triality arises from the permutations of the unit vectors
i, j, and k in 3d Euclidean space.

I should add that Tony Smith has a bunch of far-out stuff
about quaternions, octonions, Clifford algebras, triality,
the D_{4} lattice - you name it! - on his home page:


3) Tony Smith's home page, <A HREF =
"http://valdostamuseum.org/hamsmith/">http://valdostamuseum.org/hamsmith/</A>

He engages in more free association than is normally deemed
proper in scientific literature - you may raise your eyebrows at 
sentences like "the Tarot shows the Lie algebra structure of the 
D4-D5-E6 model, while the I Ching shows its Clifford algebra structure" 
- but don't be fooled; his mathematics is solid.  When it comes to the 
physics, I'm not sure I buy his theory of everything, but that's not 
unusual: I don't think I buy \emph{anyone's} theory of everything!

Let me wrap up by passing on something he told me about triality and the
exceptional groups.  In "<A HREF = "week90.html">week90</A>" I
described how you could get the Lie groups G2, F4 and E8 from triality.
I didn't know how E6 and E7 fit into the picture.  He emailed me,
saying:

\begin{quote}

"Here is a nice way:

Start with D4 = Spin(8):


\begin{verbatim}

 28 =  28  +   0  +   0  +   0  +   0  +   0  +   0
\end{verbatim}
    
Add spinors and vector to get F4:


\begin{verbatim}

 52 =  28  +   8  +   8  +   8  +   0  +   0  +   0
\end{verbatim}
    
Now, "complexify" the 8+8+8 part of F4 to get E6:


\begin{verbatim}

 78 =  28  +  16  +  16  +  16  +   1  +   0  +   1
\end{verbatim}
    
Then, "quaternionify" the 8+8+8 part of F4 to get E7:


\begin{verbatim}

133 =  28  +  32  +  32  +  32  +   3  +   3  +   3
\end{verbatim}
    
Finally, "octonionify" the 8+8+8 part of F4 to get E8:


\begin{verbatim}

248 =  28  +  64  +  64  +  64  +   7  +  14  +   7
\end{verbatim}
    
This way shows you that the "second" Spin(8) in E8
breaks down as  28 = 7 + 14 + 7
which is globally like two 7-spheres and a G2,
one S7 for left-action, one for right-action,
and a G2 automorphism group of octonions
that is needed to for "compatibility" of the two S7s.
The  3+3+3 of E7, the 1+0+1 of E6, and the 0+0+0 of F4 and D4
are the quaternionic, complex, and real analogues of the 7+14+7."

\end{quote}
When I asked him where he got this, he said he cooked it up
himself using the construction of E8 that I learned from Kostant
together with the Freudenthal-Tits magic square.  He gave
some references for the latter:

4) Hans Freudenthal, Adv. Math. 1 (1964) 143.

5) Jacques Tits, Indag. Math. 28 (1966) 223-237.

6) Kevin McCrimmon, Jordan Algebras and their applications,
Bull. AMS 84 (1978) 612-627, at pp. 620-621.
Available at <a href = "http://projecteuclid.org/DPubS?service=UI&version=1.0&verb=Display&handle=euclid.bams/1183540925">http://projecteuclid.org</a>

I would describe it here, but I'm running out of steam,
and it's easy to learn about it from his web page:

7) Tony Smith, Freudenthal-Tits magic square, 
<A HREF = "http://valdostamuseum.org/hamsmith/FTsquare.html">http://valdostamuseum.org/hamsmith/FTsquare.html</A>



\par\noindent\rule{\textwidth}{0.4pt}
<em>"I regret that it has been necessary for me in this
lecture to administer such a large dose of four-dimensional geometry.
I do not apologise, because I am not really responsible for the fact that
nature in its most fundamental aspect is four-dimensional"</em> -
Albert North Whitehead.

\par\noindent\rule{\textwidth}{0.4pt}
% </A>
% </A>
% </A>
