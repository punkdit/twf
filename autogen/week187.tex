
% </A>
% </A>
% </A>
\week{September 25, 2002 }





Okay!  Here comes the climax of our story linking q-mathematics, 
Dynkin diagrams, incidence geometry and categorification!  I'll
be amazed if you follow what I'm saying, because doing so will require
that you remember everything I've ever told you.  But I can't help 
talking about this stuff, because it's so cool.   So, please at least 
\emph{pretend} to pay attention.  I promise to ratchet things down a notch
next week.

Last time I described a bunch of things you can get from a 
Dynkin diagram:



$$

                          DYNKIN DIAGRAM
                          /             \
                         /               \
          choose a field/                 \ 
                       /                   \
                      /                     \
                     /       Weyl group      \
            SIMPLE ALGEBRAIC ----------> COXETER GROUP 
                  GROUP                       | 
                    |                         | 
             FLAG VARIETY              COXETER COMPLEX 
                     \                       /
                      \                     /
                       \                   /
                        \                 /
                         \               /
                          \             /
                           \           /
                           q-POLYNOMIAL
$$
    

If we choose any field, our Dynkin diagram gives us a group.  If we
choose the real or complex numbers we get the real and complex simple
Lie groups that physicists know and love, but it's also fun to use other
fields.  These other fields give us "simple algebraic groups",
which are not manifolds but instead algebraic varieties.

It's especially fun to choose the field F_{q} - the finite field with q
elements, where q is any power of a prime number.   The reason this is
so fun is that we can also get a group from a Dynkin diagram \emph{without}
choosing a field: the so-called Coxeter group!  Amazingly, all sorts of
formulas about this Coxeter group are special cases of formulas about
simple algebraic groups over F_{q}.  To specialize, we just set q = 1.   

In other words: the theory of simple algebraic groups over the field
with q elements is a "q-deformation" of the theory of Coxeter groups,
where q = 1.  Even better, this q-deformation is closely related to 
other q-deformations that show up in the theory of quantum groups.

This is strange and mysterious, because it seems to be saying that
Coxeter groups are simple algebraic groups over the field with one
element - but there \emph{is no field with one element!}  This mystery, 
and its relation to q-deformation, is what I find so tantalizing 
about the whole subject.  

To see the mystery play itself out before us, we need to look at the
incidence geometries having simple algebraic groups and Coxeter groups
as their symmetries.  In both cases, these incidence geometries have one
type of geometrical figure for each dot in the Dynkin diagram, and one
basic incidence relation for each edge.  

In the incidence geometry whose symmetries are a simple algebraic group
over the field F, the set of figures of a given type will be an
\emph{algebraic variety}, say X(F).  In the incidence geometry whose
symmetries are the corresponding Coxeter group, the set of figures of
this type will be a \emph{finite set}, say X.  When F_{q} is the
field with q elements, the number of points in X(F_{q}) is
finite and given by some polynomial in q.  But when we set q = 1, we get
the number of points in the set X.

To make this more clear - perhaps too clear for comfort! - I would like
to show you how to calculate all these polynomials X(F_{q}).
It's actually best to start by counting, not the set of figures of a
given type corresponding to a given dot in the Dynkin diagram, but the
set of all "maximal flags".  A maximal flag is a collection of
figures, one of each type, all incident.  We'll soon see that if we can
count these, we can count anything we want.

When we work over the field F_{q}, the set of maximal flags is
actually an algebraic variety, and the number of maximal flags is a
polynomial in q.  Last week I called this the "q-polynomial"
of our Dynkin diagram, and described how to calculate it.  In a minute
I'll say what this polynomial is in a bunch of cases.  But I can't
resist a short digression, to explain why I like this polynomial so
much!

I'm always running around trying to "categorify" everything in
sight, replacing equations by isomorphisms, numbers by finite sets, and
so on.  The reason is that we've been unconsciously
"decategorifying" mathematics for the last couple of millenia,
which is an information- destroying process, and I want to undo that
process.  For example, whenever we see a finite set, we have a tendency
to decategorify it by \emph{counting} it and just remembering its number of
elements.  Then we prove fun equations relating these numbers.  But
nowadays we know how to work directly with the finite sets and talk
about isomorphisms between them, instead of just equations between their
numbers of elements.  This gives useful extra information.

The stuff I'm talking about now is a great example.  Since the 
q-polynomial counts the number of maximal flags, it's really a 
decategorification of the variety consisting of all maximal 
flags.  But what this means is that the maximal flag variety 
is a categorification of the q-polynomial.  Using this way of 
thinking, all sorts of identities involving q-polynomials 
correspond to isomorphisms between algebraic varieties! 

Here are the q-polynomials of the classical series of Dynkin 
diagrams.  For maximum effect, this table should be read along 
with similar tables in "<A HREF = "week64.html">week64</A>" and "<A HREF = "week181.html">week181</A>".   

<UL>
<LI>
\textbf{A_{n}} The Dynkin diagram is a line of n dots:


\begin{verbatim}

                  o-------o-------o-------o-------o
\end{verbatim}
    
The Lie group is SL(n+1).   The Coxeter group is the symmetry group of 
the regular n-simplex.  This consists of all permutations of the n+1
vertices of the simplex, so it has (n+1)! elements.  The Coxeter complex
is obtained by barycentrically subdividing the surface of the n-simplex.  
The q-polynomial is the "q-factorial"


\begin{verbatim}

[n+1]! = [1] [2] ... [n+1]
\end{verbatim}
    
<LI>
\textbf{B_{n}} 
The Dynkin diagram is a line of n dots with one double edge
and an arrow indicating that the last root is shorter:


$$

                 o-------o-------o-------o====>====o 
$$
    
The Lie group is Spin(2n+1).  The Coxeter group is the symmetry group of 
an n-dimensional cube.  This group is the semidirect product of the 
permutations of the n axes and the group (Z/2)^{n} generated by the
reflections along these axes.   Thus the size of this group is the 
"double factorial" 


\begin{verbatim}

(2n)!! =  2   4  ...  2n
\end{verbatim}
    
The Coxeter complex is obtained by barycentrically subdividing the
surface of the n-dimensional cube.  The q-polynomial is the 
"q-double factorial":


\begin{verbatim}

[2n]!! = [2] [4] ... [2n]
\end{verbatim}
    
<LI>
\textbf{C_{n}}
The Dynkin diagram is a line of n dots with one double edge
and an arrow indicating that the last root is longer:


$$

                 o-------o-------o-------o====<====o 
$$
    
The Coxeter group is the symmetry group of an n-dimensional cross-polytope,
which is the obvious generalization of an octahedron to arbitrary
dimensions.  This is the exact same group as the Coxeter group of B_{n},
with the same Coxeter complex, so the q-polynomial is again the q-double
factorial:


\begin{verbatim}

[2n]!! = [2] [4] ... [2n]
\end{verbatim}
    
<LI>
\textbf{D_{n}}  The Dynkin diagram has a line of n-2 dots and then 2 more
forming a fishtail:
   

\begin{verbatim}

                                   o 
                                  /
                                 /
                 o------o-------o 
                                 \
                                  \
                                   o
\end{verbatim}
    
The Lie group is Spin(2n).  Since the Dynkin diagram is not just a
straight line of dots, it turns out the Coxeter group is not the full
symmetry group of some polytope.  Instead, it's half as big as the Weyl
group of B_{n}: it's the subgroup of the symmetries of the
n-dimensional cube generated by permutations of the coordinate axes and
reflections along \emph{pairs} of coordinate axes.  I like to call the number
of elements of this group the "half double factorial", and use
this notation for it:


\begin{verbatim}

(2n)?! = (2n)!! / 2

       =  2   4   ... (2n-2)  n
\end{verbatim}
    
The Coxeter complex is obtained from that for B_{n} by gluing together
top-dimensional simplices in pairs in a certain way to get bigger
top-dimensional simplices.  The q-polynomial is the "q-half double 
factorial":


\begin{verbatim}

[2n]?! = [2n]!! / q^n + 1

        = [2] [4] ... [2n-2] [n]
\end{verbatim}
    
</UL>
At this point I guess I should "throw a concrete life preserver to
the student drowning in a sea of abstraction", as the cruel joke
goes.  So let's actually work out some examples of these q-polynomials.
As I explained in "<A HREF = "week184.html">week184</A>", this
is easiest in base q.  The reason is that in this base, a q-integer like


$$

[5] = q^{4} + q^{3} + q^{2} + q + 1
$$
    
is written as just a string of ones like 111111, and such numbers 
are pathetically easy to multiply and divide.  So we do calculations
resembling the work of an idiot savant gone berserk:


\begin{verbatim}

A2:    [1]! = 1
A3:    [2]! = 1 x 11 = 11
A4:    [3]! = 1 x 11 x 111 = 1221
A5:    [4]! = 1 x 11 x 111 x 1111 = 1356531

B1,C1: [2]!! = 11
B2,C2: [4]!! = 11 x 1111 = 12221
B3,C3: [6]!! = 11 x 1111 x 111111 = 1357887531

D1:    [2]?! = 11 / 11 = 1
D2:    [4]?! = 11 x 1111 / 101 = 121
D3:    [6]?! = 11 x 1111 x 111111 / 1001 = 1357887531/1001 = 1356531
\end{verbatim}
    
but the results pack a considerable whallop.  For example, to count
the number of points in the maximal flag variety of Spin(7), we note 
that this group is also called B3, so its q-polynomial is [6]!!.  This 
is 1357887531 in base q, or in other words:


$$

q^{9} + 3q^{8} + 5q^{7} + 7q^{6} + 8q^{5} + 8q^{4} + 7q^{3} + 5q^{2} + 3q + 1 
$$
    
And this is the number of points of the maximal flag variety if
we work over the field with q elements!  

By the way, you may have noticed a curious coincidence in the 
above table: 


\begin{verbatim}

[4]! = [6]?!
\end{verbatim}
    
This is a spinoff of the fact that A3 and D3 are isomorphic: their
Dynkin diagrams are both just 3 dots in a row.  In "<A HREF =
"week180.html">week180</A>" I explained how this underlies
Penrose's theory of twistors.

There's a lot more we can do with these q-polynomials.  Back in "<A
HREF = "week179.html">week179</A>" and "<A HREF =
"week180.html">week180</A>" I explained some "flag
varieties" of which the maximal ones we're discussing now are
special cases.  If you give me a Dynkin diagram and a field, I will give
you a simple algebraic group G.  If you pick a subset of the dots in
this diagram, I will give you a subgroup P of G, called a
"parabolic subgroup".  The quotient G/P is called a "flag
variety".  A point in this flag variety consists of a collection of
geometrical figures of different types, one for each dot in our subset,
all incident.

The bigger the set of dots is, the smaller P is, and the bigger and
fancier the corresponding flags are.  For example, if we use
\emph{all} the dots, P is called the "Borel subgroup", and
G/P is the maximal flag variety.  On the other hand, if we use
\emph{none} of the dots, G/P is the \emph{minimal} flag variety -
just a point.  That's boring.  But if we use \emph{just one} dot, G/P
is a so-called "Grassmannian".  I listed these back in
"<A HREF = "week181.html">week181</A>", and they're really
interesting.

For example, if you give me the Dynkin diagram called D4:
  

\begin{verbatim}

                                   o 
                                  /
                                 /
                        o-------o 
                                 \
                                  \
                                   o
\end{verbatim}
    
I'll give you the group G = Spin(8,C), and I'll tell you it's the group
of conformal transformations of 6-dimensional complexified compactified
Minkowski spacetime.  If you pick out the subset consisting of just the
dot in the middle:
  

\begin{verbatim}

                                   o 
                                  /
                                 /
                        o-------x 
                                 \
                                  \
                                   o
\end{verbatim}
    
I'll tell you that G/P is the space of null lines in this spacetime.
And if you say "huh?", I'll tell you to reread "<A HREF = "week181.html">week181</A>"!

Now, for any Dynkin diagram and any subset of dots, there's a 
q-polynomial with all sorts of cool properties.  It works just like
last week: 

a) the coefficient of q^{i} in this polynomial is the number of
i-cells in the Bruhat decomposition of G/P.  Here the "Bruhat
decomposition" is a standard way of writing G/P as disjoint union
of i-cells, that is, copies of F^{i} where F is our field and i
is a natural number.

b) if the coefficient of q^{i} in this polynomial is k, the (2i)th 
homology group of G/P defined over the complex numbers is Z^{k}.

c) the degree of this polynomial is the dimension of G/P.

d) the value of this polynomial at q a prime power is the cardinality
of G/P defined over the field F_{q}.

e) the value of this polynomial at q = -1 is the Euler characteristic
of G/P defined over the real numbers.

f) the value of this polynomial at q = 1 is the Euler characteristic
of G/P defined over the complex numbers.

If we take property a) as the defining one, all the rest fall out
automagically.  By the way, the relation between the homology 
groups in part b) and the cardinalities in part d) is a special 
case of the "Weil conjectures", proved by Deligne.
For an introduction to these, try:

1) 
Robin Harshorne, Algebraic Geometry, Appendix C: The Weil conjectures,
Springer-Verlag, Berlin, 1977.

But now for the cute part: how you calculate this q-polynomial. 
It's actually really easy!  You just calculate the q-polynomial for the
whole Dynkin diagram and divide by the q-polynomial you get for the
diagram you get when you remove the dots in your subset!

So, suppose for example you got really interested in the space of 
null lines in 6d complexified compactified Minkowski spacetime:
  

\begin{verbatim}

                                   o 
                                  /
                                 /
                        o-------x 
                                 \
                                  \
                                   o
\end{verbatim}
    
The whole diagram is D4, so its q-polynomial is [8]?!   If we remove 
the dots in our subset we're left with 
  

\begin{verbatim}

                                   o 
                         

                        o
                         

                                   o
\end{verbatim}
    
that is, three copies of A1.  I never told you how to calculate the
q-polynomial for a diagram with more than one piece, but you just
multiply the q-polynomials for the pieces, so you get [2]! x [2]! x [2]!
This means the q-polynomial for our space is


\begin{verbatim}

        [8]?!          11 x 1111 x 111111 x 11111111 / 10001
   --------------   =  --------------------------------------
   [2]! [2]! [2]!                  11 x 11 x 11


                      1111   111111   11111111
                    = ---- x ------ x --------
                       11      11      10001


                    = 101 x 10101 x 1111


                    = 1020201 x 1111
 

                    = 1133443311
\end{verbatim}
    
You'll notice how all these numbers are palindromic; that comes from
Poincare duality.  We can read of all sorts of wonderful things from
the final answer, as listed above.  For example, the Euler characteristic 
of our space G/P is


\begin{verbatim}

                1+1+3+3+4+4+3+3+1 = 24
\end{verbatim}
    
The Dynkin diagram D4 is all about triality and the octonions, which are
important in superstring theory.  The number 24 plays an important role
in bosonic string theory.  Does this "coincidence" make
anything good happen?  I don't know!

That's enough for now... I'll leave off with a quote that reminds
me of these weird base q calculations.

\par\noindent\rule{\textwidth}{0.4pt}
<em>
"What's one and one and one and one and one and one and one and one
and one and one?"  

"I don't know", said Alice, "I lost count."  

"She can't do addition."</em> - Lewis Carroll, Through the Looking Glass.


\par\noindent\rule{\textwidth}{0.4pt}

% </A>
% </A>
% </A>
