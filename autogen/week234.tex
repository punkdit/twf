
% </A>
% </A>
% </A>
\week{June 12, 2006 }



Today I'd like to talk about the math of music - including 
torsors, orbifolds, and maybe even Mathieu groups.   But first, 
some movies of the n-body problem:

1) Cris Moore, The 3-body (and n-body) problem, 
<A HREF =
"http://www.santafe.edu/~moore/gallery.html">http://www.santafe.edu/~moore/gallery.html</A>

In 1993 Cris Moore discovered solutions of the gravitational n-body
problem where the particles' paths lie in a plane and trace out braids
in spacetime!  I spoke about these in "<A HREF =
"week181.html">week181</A>".

More recently, Moore and Michael Nauenberg have found solutions 
with cubic symmetry and vanishing angular momentum, and made
movies of these:

<DIV ALIGN = CENTER>
<IMG SRC = "cris_moore_cube.gif">
</DIV>

For the mathematical details, try this:

2) Cristopher Moore and Michael Nauenberg, New periodic orbits for the
n-body problem, available at <A HREF =
"http://arxiv.org/abs/math.DS/0511219">math.DS/0511219</A>.

Next, math and music.

Some of you have been in this situation.  A stranger at a party 
asks what you do.  You reluctantly admit you're a mathematician, 
expecting one of the standard responses: "Oh!  I hate math!" or 
"Oh!  <A HREF = "math_cliff.jpg">I was pretty good at math 
until</A>...."  

But instead, after a strained moment they say: "Oh!  Do you play 
an instrument too?  Isn't music really mathematical?"  

I guess it's like meeting a Martian and asking them if they like 
Arizona: an attempt to humanize something alien and threatening.  
You may not have much in common, but at least you can chat about 
red rocks.

Of course there \emph{is} something mathematical about music, and lots
of mathematicians play music.  I rarely think about music in a 
mathematical way.  But I know they have something in common: the 
transcendent beauty of pure form.  

Indeed, in the Middle Ages, music was part of a "quadrivium" of 
mathematical arts: arithmetic, geometry, music, and astronomy. 
These were studied after the "trivium" of grammar, rhetoric and 
logic.  This is why mathematicians scorn a result as "trivial" 
when it's easy to see using straightforward logic.  So when a 
result seems more profound, they should call it "quadrivial".

Try saying it sometime: "Cool!  That's quadrivial!"
It might catch on.

There are also modern applications of math to music theory.  I had 
never heard of "neo-Riemannian theory" until Tom Fiore explained it 
to me while I was visiting Chicago.  Tom is a postdoc who works on 
categorified algebraic theories, double categories and the like -
but he's also into music theory:

3) Thomas M. Fiore, Music and mathematics, available at
<A HREF = "http://www.math.uchicago.edu/~fiore/1/music.html">http://www.math.uchicago.edu/~fiore/1/music.html</A>

4) Thomas M. Fiore and Ramon Satyendra, Generalized contextual
groups, Music Theory Online 11 (2005), available at
<A HREF = "http://mto.societymusictheory.org/issues/mto.05.11.3/toc.11.3.html">http://www.math.uchicago.edu/~fiore/1/music.html</A>

The first of these is a very nice gentle introduction, suitable
both for musicians who don't know group theory and mathematicians
who don't know a triad from a tritone!

When Tom first mentioned "neo-Riemannian theory", I thought this
was some bizarre application of differential geometry to music.  
But no - we're not talking about the 19th-century mathematician 
Bernhard Riemann, we're talking about the 19th-century music 
theorist Hugo Riemann!  

Based on the work on Euler - yes, \emph{the} Euler - Hugo Riemann 
introduced diagrams called "tone nets" to study the network of 
relations between similar chords.  You can see his original 
setup here:

5) Joe Monzo, Tonnetz: the tonal lattice invented by Riemann, 
Tonalsoft: the Encyclopedia of Microtonal Music Theory,
<A HREF = "http://www.tonalsoft.com/enc/t/tonnetz.aspx">http://www.tonalsoft.com/enc/t/tonnetz.aspx</A>

6) Paul Dysart, Tonnetz: musics, harmony and donuts,
<A HREF = "http://members2.boo.net/~knuth/">http://members2.boo.net/~knuth/</A>

Apparently Riemann's ideas have caught on in a big way.  Monzo 
says that "use of lattices is endemic on internet tuning lists", 
as if they were some sort of infectious disease.  

Dysart seems more gung-ho about it all.  The "donuts" he mentions 
arise when you curl up tone nets by identifying notes that differ  
by an octave.  He has some nice pictures of them!

In neo-Riemannian theory, people like Lewin and Hyer started 
extending Riemann's ideas by using \emph{group theory} to systematize 
operations on chords.  The best easy introduction to this is 
Fiore's paper "Music and mathematics".   Here you can read about 
math lurking in the music of Elvis and the Beatles!  Or, if 
you're more of a highbrow sort, see what he has to say about 
Hindemith and Liszt's "Transcendental Etudes".  And if you 
like doughnuts and music, you'll love the section where he 
explains how Beethoven's Ninth traces out a systematic path in 
a torus-shaped tone net!  This amazing fact was discovered by
Cohn, Douthett, and Steinbach.  

(If I weren't so darn honest, I'd add that Liszt wrote the
"Transcendental Etudes" as a sequel to his popular "Algebraic
Etudes", and explain how Mozart's "eine kleine Nachtmusik" 
tours a tone net shaped like a Klein bottle.  But alas....)

Let me explain a bit about group theory and music - just 
enough to reach something really cool Tom told me.

If you're a musician, you'll know the notes in an octave go
like this, climbing up:

C, C#, D, D#, E, F, F#, G, G#, A, A#, B

until you're back to C.  If you're a mathematician, you might
be happier to call these notes

0, 1, 2, 3, 4, 5, 6, 7, 8, 9, 10, 11

and say that we're working in the group of integers mod 12, 
otherwise known as Z/12.  Let's be mathematicians today.

The group Z/12 has been an intrinsic feature of Western music
ever since pianos were built to have "equal temperament" 
tuning, which makes all the notes equally spaced in a certain 
logarithmic sense: each note vibrates at a frequency of 2^{1/12}
times the note directly below it.  

Only 7 of the 12 notes are used in any major or minor key - 
for example, C,D,E,F,G,A,B is C major and A,B,C,D,E,F,G is 
A minor.  So, as long as Western composers stuck to writing 
pieces in a single fixed key, the Z/12 symmetry was "spontaneously 
broken" by their choice of key, only visible in the freedom to 
change keys.  

But, as composers gradually started changing keys ever more
frequently within a given piece, the inherent Z/12 symmetry 
became more visible.   In the late 1800s this manifested itself
in trend called "chromaticism".   Roughly speaking, music is 
"chromatic" when it freely uses all 12 notes, but still within 
the context of an - often changing - key.  I guess Wagner and
Richard Strauss are often mentioned as pinnacles of chromaticism.

Chromaticism then led to full-fledged "twelve-tone music" 
starting with Schoenberg in the early 1900s.  This is music 
that fully exploits the Z/12 symmetry and doesn't seek to 
privilege a certain 7-element subset of notes defining a key.  
People found Schoenberg's music disturbing and dissonant at 
the time, but I find it very beautiful.  

Now comes the really exciting thing Tom told me: two other 
symmetry groups lurking in music, and a relationship between them.

First, the transposition-inversion group.  This acts as
permutations of the set Z/12.  It's generated by two 
especially nice permutations.  The first is "transposition".
This raises each note a step:
 
x |\to  x + 1 

Musicians would call this a half-step, just like physicists
measure spin in multiples of 1/2, but we're being mathematicians!
The second is "inversion".  This turns notes upside down:

x |\to  -x

The relevance of this to music is a bit less obvious: composers
like Bach and Schoenberg used it explicitly, but we'll see it
playing a subtler role, relating major and minor chords.

The transposition-inversion group has 24 elements.  Mathematicians 
call it the 24-element "dihedral group", since it consists of the 
symmetries of a regular 12-sided polygon where you're allowed 
to rotate the polygon (transposition) and also flip it over 
(inversion).  I hope you see that this geometrical picture is 
just a way of visualizing the 12 notes.

So, the transposition-inversion group obviously on the 12-element
set of notes.  But, it also acts on the 24-element set of "triads"!  

Triads are among the most basic chords in music.  Mathematically 
they are certain 3-element subsets of Z/12.   They come in two 
kinds, major and minor.  There are 12 major triads, namely 

{0,4,7}   
&nbsp;&nbsp;&nbsp;&nbsp;&nbsp;&nbsp;&nbsp;&nbsp;&nbsp;&nbsp;&nbsp;&nbsp;      
&nbsp;&nbsp;&nbsp;&nbsp;&nbsp;&nbsp;&nbsp;&nbsp;&nbsp;&nbsp;&nbsp;&nbsp;      
&nbsp;&nbsp;&nbsp;&nbsp;&nbsp;&nbsp;&nbsp;&nbsp;&nbsp;&nbsp;&nbsp;&nbsp;      
&nbsp;&nbsp;&nbsp;
     C major triad: {C,E,G}

and everything you can get from this by transposition.  If you 
invert these, you get the 12 minor triads, namely

{0,-4,-7} = {5,8,0}                   
&nbsp;&nbsp;&nbsp;&nbsp;&nbsp;&nbsp;&nbsp;&nbsp;&nbsp;&nbsp;&nbsp;&nbsp;      
&nbsp;&nbsp;&nbsp;&nbsp;&nbsp;&nbsp;&nbsp;&nbsp;&nbsp;&nbsp;&nbsp;&nbsp;      
   F minor triad: {F,Ab,C}

and everything you can get from \emph{this} by transposition.  

(Note that {0,-4,-7} = {5,8,0} because we're working mod 12
and the order doesn't matter.  I've also included the way musicians
talk about these triads, in case you care.)

Major triads sound happy; when you invert them they sound sad, 
just like an upside-down smile looks sad.  There could be some
profound truth lurking here.  A smile has a positive second 
derivative:

\begin{verbatim}
.          .
 .        .
    .  .
\end{verbatim}
    
which says that things are "looking up", while a frown has negative 
second derivative:

\begin{verbatim}
    .  .
 .        .
.          .
\end{verbatim}
    
which says that things are "looking down".  An upside-down smile
is a frown.  

(On the other hand, a backwards smile is still a smile, and a 
backwards frown is still a frown.  So, if you're a company and 
the second derivative of your profits is positive, you can say 
business is looking up - and you could still say this if time 
were reversed!)

But never mind.  We had this transposition-inversion group acting 
on our set of notes, namely Z/12.  Since tranposition and inversion
act on notes, they also act on triads.  For example, transposition 
does this:

{0,4,7} |\to  {1,5,8}               
&nbsp;&nbsp;&nbsp;&nbsp;&nbsp;&nbsp;&nbsp;&nbsp;&nbsp;&nbsp;&nbsp;&nbsp;      
&nbsp;&nbsp;&nbsp;&nbsp;&nbsp;&nbsp;&nbsp;&nbsp;&nbsp;&nbsp;&nbsp;&nbsp;      
 C major triad |\to  C# major triad

while inversion does this:

{0,4,7} |\to  {5,8,0}  
&nbsp;&nbsp;&nbsp;&nbsp;&nbsp;&nbsp;&nbsp;&nbsp;&nbsp;&nbsp;&nbsp;&nbsp;      
&nbsp;&nbsp;&nbsp;&nbsp;&nbsp;&nbsp;&nbsp;&nbsp;&nbsp;&nbsp;&nbsp;&nbsp;      
              C major triad |\to  F minor triad

So, we've got this 24-element transposition-inversion group 
acting on the 24-element set of triads!   

But here's really cool part: there's \emph{another} important 
24-element group acting on the same set!  It's easy to define 
mathematically, but it also has a musical meaning.

Mathematically, it's just the "centralizer" of the transposition-
inversion group.   In other words, it consists of all ways of
permuting triads that \emph{commute} with transposition and inversion!

Musically, it's called the "PLR" group, because it's generated
by 3 famous transformations.  

To describe these transformations, I'll need to talk about the
"bottom", "middle" and "top" note of a
triad.  If you know a wee bit of music theory this should be obvious
as long as you know I'm talking about triads in root position.  If
you're a mathematician who has never studied music theory and you
think of triads as 3-element subsets of Z/12, it might be less
obvious, since Z/12 doesn't have a nice ordering - it only has a
\emph{cyclic} ordering.  But this is enough.  The point is that major
triads are sets of the form

{n,n+4,n+7}, 

while minor triads are of the form 

{n,n+3,n+7}.  

So, we can call the note n the "bottom", the note n+3 or n+4
the "middle", and n+7 the "top".  Musicians call
them the "root", "third" and "fifth",
but let's be simple-minded mathematicians.

Okay, what are the transformations P, L, and R?  They stand for
"parallel", "leading tone change", and
"relative" - but what \emph{are} they?

Each of these transformations keeps exactly 2 of the notes 
in our triad the same.  Also, each changes major triads into
minor triads and vice versa.  These features make these 
transformations musically interesting.

The transformation "P" keeps the top and bottom notes the same.
I've now said enough for you to figure out what it does... 
at least in principle.  For example:

P: {0,4,7} |\to  {0,3,7}         
&nbsp;&nbsp;&nbsp;&nbsp;&nbsp;&nbsp;&nbsp;&nbsp;&nbsp;&nbsp;&nbsp;&nbsp;      
&nbsp;&nbsp;&nbsp;&nbsp;&nbsp;&nbsp;&nbsp;&nbsp;&nbsp;&nbsp;&nbsp;&nbsp;      
    C major triad |\to  C minor triad <br>
P: {0,3,7} |\to  {0,4,7}        
&nbsp;&nbsp;&nbsp;&nbsp;&nbsp;&nbsp;&nbsp;&nbsp;&nbsp;&nbsp;&nbsp;&nbsp;      
&nbsp;&nbsp;&nbsp;&nbsp;&nbsp;&nbsp;&nbsp;&nbsp;&nbsp;&nbsp;&nbsp;&nbsp;      
     C minor triad |\to  C major triad

The tranformation "L" turns the middle and top note into the bottom
and middle note when you start with a MAJOR triad.  It turns the 
bottom and middle note into the middle and top note when you start 
with a MINOR triad.  For example:

L: {0,4,7} |\to  {4,7,11}        
&nbsp;&nbsp;&nbsp;&nbsp;&nbsp;&nbsp;&nbsp;&nbsp;&nbsp;&nbsp;&nbsp;&nbsp;      
&nbsp;&nbsp;&nbsp;&nbsp;&nbsp;&nbsp;&nbsp;&nbsp;&nbsp;&nbsp;&nbsp;&nbsp;      
    C major triad |\to  E minor triad <br>
L: {0,3,7} |\to  {8,0,3}       
&nbsp;&nbsp;&nbsp;&nbsp;&nbsp;&nbsp;&nbsp;&nbsp;&nbsp;&nbsp;&nbsp;&nbsp;      
&nbsp;&nbsp;&nbsp;&nbsp;&nbsp;&nbsp;&nbsp;&nbsp;&nbsp;&nbsp;&nbsp;&nbsp;
&nbsp;
      C minor triad |\to  G# major triad

The transformation "R" works the other way around.   It turns the 
middle and top note into the bottom and middle note when you start 
with a MINOR triad.  And it turns the bottom and middle note into 
the middle and top note when you start with a MAJOR triad:

R: {0,4,7} |\to  {9,0,4}          
&nbsp;&nbsp;&nbsp;&nbsp;&nbsp;&nbsp;&nbsp;&nbsp;&nbsp;&nbsp;&nbsp;&nbsp;      
&nbsp;&nbsp;&nbsp;&nbsp;&nbsp;&nbsp;&nbsp;&nbsp;&nbsp;&nbsp;&nbsp;&nbsp;      
&nbsp;
   C major triad |\to  A minor triad <br>
R: {0,3,7} |\to  {3,7,10}        
&nbsp;&nbsp;&nbsp;&nbsp;&nbsp;&nbsp;&nbsp;&nbsp;&nbsp;&nbsp;&nbsp;&nbsp;      
&nbsp;&nbsp;&nbsp;&nbsp;&nbsp;&nbsp;&nbsp;&nbsp;&nbsp;&nbsp;&nbsp;&nbsp;      
    C minor triad |\to  D# major triad

Can you see why the transformations P, L, and R commute with 
transposition and inversion?   It should be easy to see that they 
commute with transposition.  Commuting with inversion means that 
if I switch the words "top" and "bottom" and also the words "major" 
and "minor" in my descriptions above, these transformations don't 
change!

You should be left wondering why P, L, and R generate the group
of \emph{all} transformations of triads that commute with transposition
and inversion - and why this group, like the transposition-inversion
group itself, has exactly 24 elements!  

It turns out some of this has a simple explanation, which has very 
little to do with the details of triads or even the 12-note scale.  
  
Imagine a scale with n equally spaced notes.   Transpositions
and inversions will generate a group with 2n elements.  Let's 
call this group G.  If you take any "sufficiently generic" chord 
in our scale, G will act on it to give a set S consisting of 2n 
different chords.  Then it's a mathematical fact that the group of 
permutations of S that commute with all transformations in G 
will be isomorphic to G!  So, it too will have 2n elements.

To explain \emph{why} this is true, I need a bit more math.

First of all, I need to define my terms.  I'm defining a chord 
to be "sufficiently generic" if no element of G maps it to itself.  
We then say G acts \emph{freely} on S.  By the way we've set 
things up, G also acts \emph{transitively} on S.  A nonempty set on which G 
acts both freely and transitively is called a "G-torsor".  You can 
read about torsors here:

7) John Baez, Torsors made easy, 
<A HREF = "http://math.ucr.edu/home/baez/torsors.html">
http://math.ucr.edu/home/baez/torsors.html</A>

They're philosophically very interesting, since they're related
to gauge symmetries in physics... but right now the only fact we
need is that any G-torsor is isomorphic to G.  So, we can identify 
S with G, with G acting by left multiplication.  

Then, it's a well-known fact that any permutation of G that
commutes with left multiplication by all elements of G must be 
given by \emph{right} multiplication by some element of G.  And
these right multiplications form a group of transformations 
that is isomorphic to G... just as we were trying to show!

In other words: the group of permutations of G has a subgroup 
isomorphic to G, namely the left translations.  It also has
another subgroup isomorphic to G, namely the right translations.
Each of these subgroups is the "centralizer" of the other.  That
is, each one consists of all permutations that commute with every 
permutation in the other one!   Fiore and Satyendra call them 
"dual groups".  

In our application to music, the first copy of G is our good old 
transposition-inversion group, while the second copy is a 
generalization of the PLR group.  Fiore and Satyendra call it the 
"generalized contextual group".

All this is indeed very general.  I don't know a similarly 
general explanation of why the operations P, L, and R succeed 
in generating all transformations that commute with transposition 
and inversion.  

I asked Tom Fiore if he and Ramon Satyendra were the first to 
show that the PRL group was the centralizer of the transposition-
inversion group.  His reply was packed with information, so 
I'll quote it:

\begin{quote}
  The initial insight about the duality between the T/I group and 
  the PLR group was at least 20 year ago.  Dual groups in the musical
  sense were introduced in David Lewin's seminal 1987 book &quot;Generalized
  Musical Intervals and Transformation Theory.&quot;  This book stimulated
  interest in neo-Riemannian theory, since Lewin recalled the
  transformations P,L, and R as objects of study.

  Major-minor duality was a concern of Hugo Riemann, a theorist from
  the second half of the 19th century.  Given his interest in duality,
  Riemann may have had some intuition about a duality between T/I and
  PLR, though it wasn't until after his death that this duality was
  formulated in algebraic terms.  An algebraic proof of the duality of
  T/I and PLR was in the thesis of Julian Hook in 2002.
   
  Ramon and I were the first to prove that the &quot;generalized contextual 
  group&quot; is dual to the T/I group acting on a set generated by an 
  arbitrary pitch-class segment satisfying the tritone condition.  
  (The tritone condition says that the inital pitch-class segment 
  contains an interval other than a tritone and unison.)  Our 
  theorem has the PLR group and major/minor triads as a special case,
  since the generalized contextual group becomes the PLR group when one
  takes the generating pitch class segment to be the three pitches of a
  major chord.  The advantage of our generalization is that one can now
  apply the PLR insight to passages that are not triadic.  There was a
  general move toward this in practice for the past decade (Childs and
  Gollin considered seventh chords rather than triads, Lewin analyzed
  instances of a non-diatonic phrase in a piano work of Schoenberg, we
  analyzed Hindemith, and so on).  Most music does not consist entirely
  of triads (e.g. late 19th century chromatic music), so the restriction
  of PLR to triads was not conclusive.
   
  We did a literature review of recent neo-Riemannian theory in Part 
  5 of our article &quot;Generalized Contextual Groups&quot;, since there have 
  been a lot of insights in the past 10 years.  One of the main
  thinkers is Rick Cohn, who came up with (among other things) a 
  nice tiling of the plane which one navigates using P,L, and R 
  (Richard Cohn, Neo-Riemannian operations, parsimonious trichords, 
  and their Tonnetz representations, Journal of Music Theory, 1997).  
  It is quite geometric. 

\end{quote}
    
You read more about these matters here... I'll list these references 
in the order Tom mentions them:

8) David Lewin, Generalized Musical Intervals and Transformations,
Yale University Press, New Haven, Connecticut, 1987.

9) Julian Hook, Uniform Triadic Transformations, Ph.D. thesis, Indiana 
University, 2002.

10) Adrian P. Childs, Moving beyond neo-Riemannian triads: exploring 
a transformational model for seventh chords, Journal of Music
Theory 42/2 (1998), 191-193.

11) Edward Gollin, Some aspects of three-dimensional Tonnetze,
Journal of Music Theory 42/2 (1998), 195-206.

12) Richard Cohn, Neo-Riemannian operations, parsimonious 
trichords, and their "Tonnetz" representations, Journal of 
Music Theory 41/1 (1997), 1-66.

13) David Lewin, Transformational considerations in Schoenberg's 
Opus 23, Number 3, preprint. 

In fact, the notion of "torsor" pervades the work of David
Lewin, but not under this name - Lewin calls it a "general
interval system".  Stephen Lavelle noticed the connection to
torsors in 2005:

14) Stephen Lavelle, Some formalizations in musical set theory,
June 3, 2005, available at <A HREF = "http://www.maths.tcd.ie/~icecube/lewin.pdf">http://www.maths.tcd.ie/~icecube/lewin.pdf</A>
and <A HREF = "http://www.maths.tcd.ie/~icecube/lewin.ps">http://www.maths.tcd.ie/~icecube/lewin.ps</A>

Unfortunately the music theorists seem not to have set up 
an "arXiv", so some of their work is a bit hard to find.
For example, all of Volume 42 Issue 2 of the Journal of Music 
Theory is dedicated to neo-Riemannian theory, but I don't
think it's available online.  Luckily, the music theorists have 
set up some free online journals, like this:

15) Music Theory Online, <A HREF = "http://mto.societymusictheory.org/">http://mto.societymusictheory.org/</A>

and this one has links to others.  The Society for Music Theory 
also has online resources including a nice bibliography on the 
basics of music theory:

16) Society for Music Theory, Fundamentals of music theory,
selected bibliography, 
<A HREF = "http://societymusictheory.org/index.php?pid=37">http://societymusictheory.org/index.php?pid=37</A>

Now let me turn up the math level a notch....

If you're the right sort of mathematician, you'll have noticed by
now that we're doing some fun stuff starting with the abelian
group A = Z/12.   First we're forming the group G consisting of all
"affine transformations" of A.  These are the transformations that 
preserve all these operations:

(x,y) |\to  cx + (1-c)y                

where c is an integer.   For A = Z/n, the group of affine
transformations has the transposition-inversion group as a
subgroup.  The whole affine group has 48 elements, but for 
now we only keep this subgroup with 24 elements.  Call it G.


Then, we're saying that we can take any "sufficiently generic" 
subset of A, hit it with all elements of G, and get a G-torsor, 
say S.  G is then seen as a subgroup of the group of permutations 
of S, and the centralizer of this subgroup is again isomorphic to 
G.

You may be more familiar with affine transformations on a vector 
space, where we get to use any real number for c.   Then 

cx + (1-c)y       

describes the line through x and y, so you can say that affine 
transformations are those that preserve lines.  Vector spaces are 
R-modules for R the reals, while abelian groups are R-modules for 
R the integers.  The concept of "affine transformations" of an 
R-module works pretty much the same way whenever R is any 
commutative ring.   And, indeed, everything I just said in the last
paragraph works if we let A be an R-module for any commutative ring 
R.   

So, there's some very simple nice abstract stuff going on here:
we're taking an abelian group A, looking at a subgroup G of its
affine transformations, and seeing that sufficiently generic
subsets of A give rise to G-torsors!

These are nice examples of G-torsors, since nobody is likely to 
accidentally confuse them with the group G.  If you read my webpage
on torsors, you'll see it's often easy to mix up a G-torsor with 
the group G itself.

In fact, I just committed this sin myself!   The set of notes is 
not naturally an abelian group until we pick an origin - a place 
for the chromatic scale to start.  It's really just an A-torsor, 
where A is the abelian group generated by transposition.  

So, there lots of torsors lurking in music....

The pretty math I've just described only captures a microscopic 
portion of what makes music interesting.  It doesn't, for example,
have anything to say about what makes some intervals more dissonant
than others.  As Pythagoras noticed, simple frequency ratios like 
3/2 or 4/3 make for less dissonant chords than gnarly fractions 
like 1259/723.  The equal tempered tuning system, where the basic
frequency ratio is 2^{1/12}, would have made Pythagoras roll in 
his grave!   Advocates of other tuning systems say these irrational 
frequency ratios are driving us crazy, making wars break out and 
plants wilt - but there's an unavoidable conflict between the desire 
for simple ratios and the desire for evenly spaced notes, built into 
the fabric of mathematics and music.  Every tuning system is thus a
compromise.  I would like to understand this better; there's bound 
to be a lot of nice number theory here.

To study different tuning systems in a unified way, one first step
is replace the group Z/12 by a continuous circle.  Points on this
circle are "frequencies modulo octaves", since for many - though 
certainly not all - purposes it's good to consider two notes 
"the same" if they differ by an octave.  Mathematically this circle 
is R^{+}/2, namely the multiplicative group of positive real numbers 
modulo doubling.  As a group, it's isomorphic to the usual circle
group, U(1). 

This "pitch class circle" plays a major role in the work of Dmitri 
Tymoczko, a composer and music theorist from Princeton, who emailed 
me after I left a grumpy comment on the discussion page for this 
fascinating but slightly obscure article:

17) Wikipedia, Musical set theory,
<A HREF = "http://en.wikipedia.org/wiki/Musical_set_theory">http://en.wikipedia.org/wiki/Musical_set_theory</A>

He's recently been working on voice leading and orbifolds.  They're 
related topics, because if you have a choir of n indistinguishable 
angels, each singing a note, the set of possibilities is:

T^{n}/S_{n}

where T^{n} is the n-torus - the product of n copies of the
pitch class circle - and S_{n} is the permutation group,
acting on n-tuples of notes in the obvious way.  This quotient is not
usually a manifold, because it has singularities at certain points
where more than one voice sings the same note.  But, it's an
\emph{orbifold}.  This kind of slightly singular quotient space is
precisely what orbifolds were invented to deal with.

Tymoczko is coming out with an article about this in Science 
magazine.  For now, you can learn more about the geometry of 
music by playing with his "ChordGeometries" software:

18) Dmitri Tymoczko, ChordGeometries,   
<A HREF = "http://music.princeton.edu/~dmitri/ChordGeometries.html">http://music.princeton.edu/~dmitri/ChordGeometries.html</A>

As for "voice leading", let me just quote his explanation, 
suitable for mathematicians, of this musical concept:

\begin{quote}
  BTW, if you're writing on neo-Riemannian theory in music, it 
  might be helpful to keep the following basic distinction in 
  mind.  There are chord progressions, which are essentially 
  functions from unordered chords to unordered chords (e.g. the 
  chord progression (function) that takes C major to E minor).

  Then there are voice leadings, which are mappings from the notes 
  of one chord to the notes of the other E.g. &quot;take the C in a C 
  major triad and move it down by semitone to the B.&quot;  This voice 
  leading can be written: 
(C, E, G) |&rarr; (B, E, G).

  This distinction is constantly getting blurred by neo-Riemannian 
  music theorists.  But to really understand &quot;neo-Riemannian 
  chord progressions&quot; you have to be quite clear about it.

  To form a generalized neo-Riemannian chord progression, start 
  with an ordered pair of chords, say (C major, E minor).  Then 
  apply all the transpositions and inversions to this pairs, 
  producing (D major, F# minor), (C minor, Ab major), etc.  The 
  result is a function that commutes with the isometries of the 
  pitch class circle.  As a result, it identifies pairs of chords 
  that can be linked by exactly similar collections of voice 
  leading motions.

  For example, I can transform C major to E minor by moving C down 
  by semitone to B.

  Similarly, I can transform D major to F# minor by moving D down 
  by semitone to C#.

  Similarly, I can transform C minor to Ab major by moving G up to 
  Ab.

  This last voice leading, 
(C, Eb, G) |&rarr; (C, Eb, Ab) 
is just an inversion (reflection) of the voice leading 
(C, E, G)| &rarr; (B, E, G).  
  As a result it moves one note up by semitone, rather than moving
  one note down by semitone.

  More generally: if you give me <em>any</em> voice leading between C 
  major and E minor, I can give you an exactly analogous voice 
  leading between D major and F# minor, or C minor and Ab major, 
  etc.  So &quot;neo-Riemannian&quot; progressions identify a class of 
  <em>harmonic</em> progressions (functions between unordered collections
  of points on the circle) that are interesting from a <em>voice 
  leading</em> perspective.  (They identify pairs of chord progressions 
  that can be linked by the same voice leadings, to within rotation 
  and reflection.)
\end{quote}
    

You can learn more about this here:

19) Dmitri Tymoczko, Scale theory, serial theory, and voice leading,
available at <A HREF = "http://music.princeton.edu/~dmitri/scalesarrays.pdf">http://music.princeton.edu/~dmitri/scalesarrays.pdf</A>

I'd like to conclude tonight's performance with a 
"chromatic fantasy" - some wild ideas that you shouldn't
take too seriously, at least as far as music theory goes.  In this
rousing finale, I'll list some famous subgroups of the permutations of
a 12-element set.  They may not be relevant to music, but I can't
resist mentioning them and hoping somebody dreams up an application.

So far I've only mentioned two: the cyclic or "transposition" group, 
Z/12, and the dihedral or "transposition/inversion" group with 24 
elements.  These are motivated by thinking of Z/12 as a discrete
analogue of a circle and considering either just its rotations, or 
rotations together with reflections.  But, mathematically, it's
nice to loosen up this rigid geometry and consider \emph{projective}
transformations of a circle, now viewed as a line together with a 
point at infinity - a "projective line".

Indeed, the group Z/11 becomes a field with 11 elements if we multiply
as well as add mod 11.  If we throw in a point at infinity, we get a 
projective line with 12 elements.  It looks just like our circle of 12 
notes.  But now we see that the group PGL(2,Z/11) acts on this projective 
line in a natural way.  This group consists of invertible 2\times 2 matrices 
with entries in Z/11, mod scalars.  People call it PGL(2,11) for short.

So, PGL(2,11) acts on our 12-element set of notes.  And, it's a 
general fact for any field F that PGL(2,F) acts on the corresponding
projective line in a "triply transitive" way.  In other words, given
any ordered triple of distinct points on the projective line, we can 
find a group element that maps it to any \emph{other} ordered triple of
distinct points.  

Even better, the action is "sharply" triply transitive, meaning
there's \emph{exactly one} group element that does the job!  

This lets us count the elements in PGL(2,11).  Since we can find 
exactly one group element that maps our favorite ordered triple of 
distinct elements to any other, we just need to count such triples,
and there are 

12 \times  11 \times  10 = 1320

of them - so this is the size of PGL(2,11).

This may be too much symmetry for music, since this group carries
\emph{any} three-note chord to any other, not just in the sense of 
chord progressions but in the sense of voice leadings.  Still,
it's cute.

We might go further and look for a quadruply transitive group of
permutations of our 12-element set of notes - in other words, one
that maps any ordered 4-tuple of distinct notes to any other.

But if we do, we'll run smack dab into MATHIEU GROUPS!

Here's an utterly staggering fact about reality.  Apart from the group
of \emph{all} permutations of an n-element set
and the group of \emph{even}
permutations of an n-element set, 
there are only FOUR groups of permutations that are k-tuply
transitive for k > 3.  Here they are:

<UL>
<LI>
The Mathieu group M_{11}.  This is a quadruply transitive group 
of permuations of an 11-element set - and sharply so!  It has

11 \times  10 \times  9 \times  8 = 7920 

elements.

<LI>
The Mathieu group M_{12}.  This is a quintuply transitive group 
of permutations of a 12-element set - and sharply so!  It has 

12 \times  11 \times  10 \times  9 \times  8 = 95,040

elements.  

<LI>
The Mathieu group M_{23}.  This is a quadruply transitive group
of permutations of a 23-element set - but not sharply so.  It has

23 \times  22 \times  21 \times  20 \times  48 = 10,200,960

elements.   As you can see, 48 group elements carry any distinct
ordered 4-tuple to any other. 

<LI>
The Mathieu group M_{24}.  This is a quintuply transitive group
of permutations of a 24-element set - but not sharply so.  It has

24 \times  23 \times  22 \times  21 \times  20 \times  48 = 244,823,040

elements.   As you can see, 48 group elements carry any distinct
ordered 4-tuple to any other. 
</UL>

These groups all arise as symmetries of certain discrete geometries
called Steiner systems.  An "S(L,M,N) Steiner system" is a
set of N "points" together with a collection of
"lines", such that each line contains M points, and \emph{any}
set of L points lies on a unique line.  The symmetry group of a
Steiner system consists of all permutations of the set of points that
map lines to lines.  It turns out that:

<UL>
<LI>
There is a unique S(5,6,12) Steiner system, and the Mathieu group
M_{12} is its symmetry group.  The stabilizer group of any point 
is isomorphic to M_{11}.

<LI>
There is a unique S(5,8,24) Steiner system, and the Mathieu group
M_{24} is its symmetry group.  The stabilizer group of any point 
is isomorphic to M_{23}.
</UL>

So, the group M_{12} could be related to music if there were a 
musically interesting way of taking the chromatic scale and choosing 
6-note chords such that any 5 notes lie in a unique chord.  I can't 
imagine such a way - most of these chords would need to be 
wretchedly dissonant.  Another way to put the problem is that such 
a big group of permutations would impose more symmetry on the set
of chords than I can imagine my ears hearing.  It's like those 
grand unified theories that posit symmetries interchanging particles 
that look completely different.   They could be true, but they've 
got their work cut out for them.

Luckily, the Mathieu groups appear naturally in other contexts - 
wherever the numbers 12 and 24 cast their magic spell over mathematics!  
For example, M_{24} is related to the 24-dimensional Leech lattice, 
and M_{12} can be nicely described in terms of 12 equal-sized balls 
rolling around the surface of another ball of the same size.  See 
"<A HREF = "week20.html">week20</A>" for more on this - and the book by Conway and Sloane cited 
there for even more.  

For a pretty explanation of M_{24}, also try this:

20) Steven H. Cullinane, Geometry of the 4 \times  4 square, 
<A HREF = "http://finitegeometry.org/sc/16/geometry.html">http://finitegeometry.org/sc/16/geometry.html</A>

For explanations of both M_{24} and M_{12}, try this:

21) Peter J. Cameron, Projective and Polar Spaces, QMW Math Notes
13, 1991.  Also available at <A HREF = "http://www.maths.qmul.ac.uk/~pjc/pps/">http://www.maths.qmul.ac.uk/~pjc/pps/</A>
Chapter 9: The geometry of the Mathieu groups, available at 
<A HREF = "http://www.maths.qmul.ac.uk/~pjc/pps/pps9.pdf">http://www.maths.qmul.ac.uk/~pjc/pps/pps9.pdf</A>

It would be fun to dream up more relations between incidence geometry
and music theory.  Could Klein's quartic curve play a role?  Remember
from "<A HREF = "week214.html">week214</A>", "<A HREF =
"week215.html">week215</A>" and "<A HREF =
"week219.html">week219</A>" that this 3-holed torus can be nicely
tiled by 24 regular heptagons:

<DIV ALIGN = CENTER>
<IMG SRC = Klein168.gif>
</DIV>

Its orientation-preserving symmetries
form the group PSL(2,7), which consists of all 2x2 matrices with
determinant 1 having entries in Z/7, modulo scalars.  This group has
24 \times  7 = 168 elements.  Since there are 7 notes in a major or
minor scale, and 24 of these scales, it's hard to resist wanting to
think of each heptagon as a scale!

Indeed, after I mentioned this idea to Dmitri Tymoczko, he said
that David Lewin and Bob Peck have written about related topics.  

Alas, the heptagonal tiling of Klein's quartic has a total of 56 
vertices, not a multiple of 12, so there's no great way to think 
of the vertices as notes.  But, it has 84 = 7 \times  12 edges, so 
maybe the edges are labelled by notes and each note labels 7 edges.

Unlike some groups I mentioned earlier, PSL(2,7) is not a transitive
subgroup of the permutations of a 12-element set.  And while PSL(2,7) has 
lots of 12-element subgroups, these are not cyclic groups but 
instead copies of A_{4}.  These facts put some further limitations 
on any crazy ideas you might try.

On the bright side, mathematically if not musically, there is a 
fascinating way to embed PSL(2,7) into the Mathieu group M_{24},
which can be described by getting M_{24} to act on the
set of 24 heptagons in the Klein quartic:

22) David Richter, How to make the Mathieu group M_{24},
<A HREF = "http://homepages.wmich.edu/~drichter/mathieu.htm">http://homepages.wmich.edu/~drichter/mathieu.htm</A>

He works in the Poincar&eacute; dual picture, where the Klein quartic
is tiled by 56 triangles, but that's no big deal.

By the way, in "<A HREF = "week79.html">week79</A>" I explained how
PSL(2,F) acts on the projective line over the field F; the same thing
works for PGL(2,F).  I also passed on some interesting facts mentioned
by Bertram Kostant, which relate PSL(2,5), PSL(2,7) and PSL(2,11) to
the symmetry groups of the tetrahedron, cube/octahedron and
dodecahedron/icosahedron.  Kostant put these together to give a nice
description of the buckyball!

Kepler would be pleased.  But, he'd be happier if we could find
the music of the spheres lurking in here, too.
\par\noindent\rule{\textwidth}{0.4pt}
<B>Addenda:</B>  This week's issue provoked more discussion than any in 
recent history!  You can read a lot on
<A HREF = "http://groups.google.com/group/sci.physics.research/browse_frm/thread/e86672156f1477ea/d572967a25665fba">sci.math.research</A>.
Here are some comments from Dave Rusin, David Corfield, Mike
Stay, Dmitri Tymoczko, Cris Moore, Robert Israel, Noam Elkies, Stephen 
Lavelle, and <a href = "http://www.youtube.com/watch?v=-tPuwCUulBQ">Steve 
Lubin</a>.

Dave Rusin explained the logic behind having 12 notes in the chromatic
scale.  David Corfield mentioned a book on topos theory in music, and
a paper by Noam Elkies on Mathieu groups.  Mike Stay pointed out William
Sethares' work on how the timbre of an instrument affects which scales
sound good.  Dmitri Tymoczko had more comments on this issue.  Cris Moore
mentioned an interesting microtonal composer named Easley Blackwood.
Robert Israel pointed out an unusual fact about Riemann and Einstein.
Noam Elkies explained what David Lewin was trying to do with PSL(2,7) in
music theory.  And Stephen Lavelle gave some more references on torsors
and topoi in music, and said more about the origin of the 12-note scale.

So, here we go!  Dave Rusin wrote:



% parser failed at source line 1117
